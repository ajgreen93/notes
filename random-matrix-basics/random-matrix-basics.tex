\documentclass{article}

% Packages
\usepackage[utf8]{inputenc} % allow utf-8 input
\usepackage[T1]{fontenc}    % use 8-bit T1 fonts
\usepackage{booktabs}       % professional-quality tables
\usepackage{nicefrac}       % compact symbols for 1/2, etc.
\usepackage{microtype}      % microtypography
% \usepackage{times}             % times font
\usepackage{mathrsfs}      % Added by Alden, for script font.
\usepackage{ragged2e}     % Added by Alden, for no indent.
\usepackage{parskip}        % Added by Alden, for skips between paragraphs.

\usepackage[round]{natbib}
\usepackage{amssymb,amsmath,amsthm,bbm}
\usepackage[margin=1in]{geometry}
\usepackage{verbatim,float,url,dsfont}
\usepackage{graphicx,subcaption,psfrag} % Alden changed subfigure to subcaption
\usepackage{algorithm,algorithmic}
\usepackage{mathtools}
\usepackage[shortlabels]{enumitem}       % Alden added shortlabels option
\usepackage[colorlinks=true,citecolor=blue,urlcolor=blue,linkcolor=blue]{hyperref}
\usepackage{multirow}

% Theorems and such
\newtheorem{theorem}{Theorem}
\newtheorem{lemma}{Lemma}
\newtheorem{corollary}{Corollary}
\newtheorem{proposition}{Proposition}
\theoremstyle{definition}
\newtheorem{remark}{Remark}
\newtheorem{definition}{Definition}
\newtheorem{example}{Example} % Added by Alden
\newtheorem{conjecture}{Conjecture} % Added by Alden

% Assumption
\newtheorem*{assumption*}{\assumptionnumber}
\providecommand{\assumptionnumber}{}
\makeatletter
\newenvironment{assumption}[2]{
  \renewcommand{\assumptionnumber}{Assumption #1#2}
  \begin{assumption*}
  \protected@edef\@currentlabel{#1#2}}
{\end{assumption*}}
\makeatother

% Widebar
\makeatletter
\newcommand*\rel@kern[1]{\kern#1\dimexpr\macc@kerna}
\newcommand*\widebar[1]{%
  \begingroup
  \def\mathaccent##1##2{%
    \rel@kern{0.8}%
    \overline{\rel@kern{-0.8}\macc@nucleus\rel@kern{0.2}}%
    \rel@kern{-0.2}%
  }%
  \macc@depth\@ne
  \let\math@bgroup\@empty \let\math@egroup\macc@set@skewchar
  \mathsurround\z@ \frozen@everymath{\mathgroup\macc@group\relax}%
  \macc@set@skewchar\relax
  \let\mathaccentV\macc@nested@a
  \macc@nested@a\relax111{#1}%
  \endgroup
}
\makeatother

% Min and max
\newcommand{\argmin}{\mathop{\mathrm{argmin}}}
\newcommand{\argmax}{\mathop{\mathrm{argmax}}}
\newcommand{\minimize}{\mathop{\mathrm{minimize}}}
\newcommand{\st}{\mathop{\mathrm{subject\,\,to}}}
\DeclareMathOperator*{\esssup}{ess\,sup} % Added by Alden

% Shortcuts
\def\R{\mathbb{R}}
\def\C{\mathbb{C}}
\def\E{\mathbb{E}}
\def\P{\mathbb{P}}
\def\T{\mathsf{T}}
\def\Cov{\mathrm{Cov}}
\def\Var{\mathrm{Var}}
\def\half{\frac{1}{2}}
\def\tr{\mathrm{tr}}
\def\df{\mathrm{df}}
\def\dim{\mathrm{dim}}
\def\col{\mathrm{col}}
\def\row{\mathrm{row}}
\def\nul{\mathrm{null}}
\def\rank{\mathrm{rank}}
\def\nuli{\mathrm{nullity}}
\def\spa{\mathrm{span}}
\def\sign{\mathrm{sign}}
\def\supp{\mathrm{supp}}
\def\diag{\mathrm{diag}}
\def\aff{\mathrm{aff}}
\def\conv{\mathrm{conv}}
\def\dom{\mathrm{dom}}
\def\hy{\hat{y}}
\def\hf{\hat{f}}
\def\hmu{\hat{\mu}}
\def\halpha{\hat{\alpha}}
\def\hbeta{\hat{\beta}}
\def\htheta{\hat{\theta}}
\def\cA{\mathcal{A}}
\def\cB{\mathcal{B}}
\def\cD{\mathcal{D}}
\def\cE{\mathcal{E}}
\def\cF{\mathcal{F}}
\def\cG{\mathcal{G}}
\def\cK{\mathcal{K}}
\def\cH{\mathcal{H}}
\def\cI{\mathcal{I}}
\def\cL{\mathcal{L}}
\def\cM{\mathcal{M}}
\def\cN{\mathcal{N}}
\def\cP{\mathcal{P}}
\def\cS{\mathcal{S}}
\def\cT{\mathcal{T}}
\def\cW{\mathcal{W}}
\def\cX{\mathcal{X}}
\def\cY{\mathcal{Y}}
\def\cZ{\mathcal{Z}}

%%% Begin Alden's additions
\newcommand{\Ebb}{\mathbb{E}}
\newcommand{\Pbb}{\mathbb{P}}
\newcommand{\dotp}[2]{\langle #1, #2 \rangle}
\newcommand{\wt}[1]{\widetilde{#1}}
\newcommand{\wh}[1]{\widehat{#1}}
\newcommand{\mc}[1]{\mathcal{#1}}
\newcommand{\Reals}{\mathbb{R}} % Same thing as Ryan's \R
\newcommand{\Rd}{\Reals^d}
\newcommand{\wb}[1]{\widebar{#1}}
\newcommand{\floor}[1]{\left\lfloor #1 \right\rfloor}
\newcommand{\ceil}[1]{\left\lceil #1 \right\rceil}
\newcommand{\1}{\mathbf{1}}
\newcommand{\bj}{{\bf j}}
\newcommand{\restr}[2]{\ensuremath{\left.#1\right|_{#2}}}
\newcommand{\TV}{\mathrm{TV}}

\DeclareFontFamily{U}{mathx}{\hyphenchar\font45}
\DeclareFontShape{U}{mathx}{m}{n}{<-> mathx10}{}
\DeclareSymbolFont{mathx}{U}{mathx}{m}{n}
\DeclareMathAccent{\wc}{0}{mathx}{"71}


%%% End Alden's Additions
\usepackage{lmodern}
\usepackage{xcolor}
\usepackage{diagbox}
\usepackage{xr-hyper}

\newcommand{\bx}{\boldsymbol{x}}
\newcommand{\bz}{\boldsymbol{z}}
\newcommand{\bv}{\boldsymbol{v}}
\newcommand{\bw}{\boldsymbol{w}}
\newcommand{\bq}{\boldsymbol{q}}
\newcommand{\bX}{\boldsymbol{X}}
\newcommand{\bbeta}{\boldsymbol{\beta}}
\newcommand{\bSigma}{\boldsymbol{\Sigma}}
\newcommand{\bTheta}{\boldsymbol{\Theta}}
\newcommand{\bS}{\boldsymbol{S}}
\newcommand{\bB}{\boldsymbol{B}}
\newcommand{\bQ}{\boldsymbol{Q}}
\newcommand{\bP}{\boldsymbol{P}}
\newcommand{\bR}{\boldsymbol{R}}
\newcommand{\bI}{\boldsymbol{I}}
\newcommand{\bZ}{\boldsymbol{Z}}

\title{ {\bf The Basics of Random Matrix Theory} }

\begin{document}
	
\maketitle
\RaggedRight

\begin{abstract}
	This document records some elementary background on random matrix theory for sample covariance matrices.
\end{abstract}

\section{Introduction}

Suppose random vectors $\bx_1,\ldots,\bx_n \in \Reals^p$ are independently sampled from a Normal distribution with mean $0$ and covariance $\bSigma$. The sample covariance matrix is
$$
\bS = \frac{1}{n}\sum_{i = 1}^{n} \bx_i \bx_i^{\top} = \frac{1}{n}\bX \bX^{\top},
$$ 
where $\bX \in \Reals^{n \times p}$ is matrix with rows $\bX_{i\cdot} = \bx_i$. Random matrix theory describes the asymptotic behavior of the sample eigenvalues $\lambda_i(\bS) = \lambda_i(\bS)$ and eigenvectors $\bv_i = \bv_i(\bS)$ of $\bS$, in the \emph{proportional asymptotics} limit $n,p \to \infty, p/n \to \gamma \in (0,\infty)$. 

\subsection{Stieltjes transform}
The Stieltjes transform is a key analytical device used to prove asymptotic convergence of spectral distributions of random matrices. The {\bf Stieltjes transform} of a measure $\mu$ is defined for $z \in \mathbb{C}\setminus\supp(\mu)$ as
$$
m_{\mu}(z) := \int \frac{1}{t - z} \,d\mu(t), \quad m_{\mu}(z) \in C_{+}.
$$
The measure $\mu$ can be recovered from its Stieltjes transform via the {\bf Stieltjes inversion formula}. At a given $\tau \in \supp(\mu)$, if $\lim_{\varepsilon \to 0} \Im m_{\mu}(\tau + \varepsilon i)$ exists then $m_{\mu}$ is continuous at $t$, with density $f_{\mu}(\tau) = \frac{1}{\pi}\lim_{\varepsilon} \Im m_{\mu}(\tau + \varepsilon i)$. To see this:
\begin{align*}
	\lim_{\varepsilon \to 0}\Im m_{\mu}(\tau + \varepsilon i) 
	& = 
	\lim_{\varepsilon \to 0}\int \Im(\frac{1}{t - (\tau + \varepsilon i)} \,d\mu(t) \\
	& = \lim_{\varepsilon \to 0} \int \frac{\varepsilon}{(t - \tau)^2 + \varepsilon^2} \,d\mu(t) \\
	& = \frac{1}{\pi}f(\tau).
\end{align*} 

\section{Distribution of eigenvalues}
Let $\hat{H}_n(\tau)$ represent the empirical spectral distribution of the eigenvalues of $\bSigma$:
$$
\hat{H}_n(t) := \frac{1}{p} \sum_{i = 1}^{p} \1\{\lambda_i(\bSigma) \leq t\}.
$$
Suppose that $\hat{H}_n(t)$ converges weakly to a limiting spectral distribution $H(t)$ as $p \to \infty$. Then the spectral distribution $\hat{F}_n(t)$ of the sample covariance $\bS$ also converges to continuum limit $F_{\gamma,H}(t)$, which we will call a {\bf generalized Marchenko-Pastur distribution}. Random matrix theory provides the analytical tools necessary to prove this result and describe the resulting continuum limit. We give the result and sketch the proof.

\subsection{The fixed point equation}
Let $m_{\gamma,H}$ be the Stieltjes transform of the limiting generalized Marchenko-Pastur distribution $F_{\gamma,H}$. The companion transform is $v_{\gamma,H} := \gamma m_{\gamma,H}(z) - \frac{(1 - \gamma)}{z}$. It can be shown, via the leave-one-out approach, that the companion transform satisfies the self-consistency equation
\begin{equation}
	\label{eqn:silverstein}
	z = -\frac{1}{v(z)} + \gamma \int \frac{t}{1 + tv(z)} \,dH(t), \quad \textrm{for all $z \in \C_{+}$.}
\end{equation}

\subsection{The companion transform}
To begin, let $\bB = \bS^{\top} = \frac{1}{n} \bX^{\top} \bX \in \Reals^{n \times n}$. We will assume without loss of generality that $\bSigma$ is diagonal with elements $\tau_1,\ldots,\tau_p$, and write $\bB = \bQ \bSigma \bQ$, where $\bQ_{ij} \sim N(0,1/n)$, all independent. 

The spectral distribution of $\bB$ is 
$$
\bar{F}_n(t) = \frac{1}{n}\sum_{i = 1}^{n} \1\{\lambda_i(\bB) \leq t\} = \gamma \hat{F}_n(t) + \max\{(1 - \gamma),0\} \1\{t \leq 0\},
$$
so the Stieltjes transform of $\bB$ is 
$$
v_n(z) = \frac{1}{n} \tr\Big((\bB - z\bI)^{-1}\Big) = \gamma m_n(z) - \frac{(1 - \gamma)}{z}.
$$
The transform $v_n$ is called the {\bf companion transform} of $m_n$. We will sketch an argument showing that the companion transform converges in the proportional asymptotic limit. To begin, we make the following simplifying approximations:
\begin{align*}
	\bq_{(j)}^{\top} (\bB - z\bI)^{-1} \bq_{(j)} & \approx \mathbb{E}[\bq_{(j)}^{\top} (\bB - z\bI)^{-1} \bq_{(j)}] \\
	\bq_{(j)}^{\top} (\bB_j - z\bI)^{-1} \bq_{(j)} & \approx \mathbb{E}[\bq_{(j)}^{\top} (\bB_j - z\bI)^{-1} \bq_{(j)}] \\
	v_n(z) := \frac{1}{n}\tr(\bB - z \bI)^{-1} & \approx \mathbb{E} v_n(z) \\
	\mathbb{E}{\frac{1}{n}\tr(\bB - z \bI)^{-1}} & \approx \mathbb{E}{\frac{1}{n}\tr(\bB_j - z \bI)^{-1}}.
\end{align*}


\subsection{The high-level argument}
Let $\bq_{(j)} \in \Reals^{n}$ be the $j$th column of $\bQ$ and notice that 
$$
\bB = \sum_{j = 1}^{p} \tau_j \bq_{(j)} \bq_{(j)}^{\top}.
$$
The key argument involves the concentration of terms of the form $\bq_{(j)}^{\top} (\bB - z\bI)^{-1} \bq_{(j)}$. Via the Sherman-Morrison formula, 
$$
(\bB - z\bI)^{-1} 
= (\bB_j -  z \bI + \tau_j \bq_{(j)} \bq_{(j)}^{\top})^{-1} 
= (\bB_j - z\bI)^{-1} - \frac{\tau_j (\bB_j - z\bI)^{-1} \bq_{(j)} \bq_{(j)}^{\top} (\bB_j - z\bI)^{-1}}{1 + \tau_j \bq_{(j)}^{\top}(\bB_j - z\bI)^{-1}\bq_{(j)}}
$$
hence,
\begin{align*}
\bq_{(j)}^{\top} (\bB - z\bI)^{-1} \bq_{(j)} 
& = \bq_{(j)}^{\top} (\bB_j - z\bI)^{-1} \bq_{(j)} - \frac{\tau_j}{1 + \tau_j \bq_{(j)}^{\top}(\bB_j - z\bI)^{-1}\bq_{(j)}} (\bq_{(j)}^{\top} (\bB_j - z\bI)^{-1} \bq_{(j)})^2 \\
& = \frac{\bq_{(j)}^{\top} (\bB_j - z\bI)^{-1} \bq_{(j)}}{1 + \tau_j \bq_{(j)}^{\top}(\bB_j - z\bI)^{-1}\bq_{(j)}}.
\end{align*}
Since $q_{(j)}$ and $B_j$ are independent,
$$
\mathbb{E}[\bq_{(j)}^{\top} (\bB_j - z\bI)^{-1} \bq_{(j)}|\bX_{(-j)}] =  \frac{1}{n}\tr(\bB_j - z\bI)^{-1}.
$$
Thus, using two of our simplifying approximations, we have that
$$
\bq_{(j)}^{\top} (\bB_j - z\bI)^{-1} \bq_{(j)}  \approx \mathbb{E}[\bq_{(j)}^{\top} (\bB_j - z\bI)^{-1} \bq_{(j)}] = \E[\frac{1}{n}\tr(\bB_j - z\bI)^{-1}]  \approx \mathbb{E}v_n(z),
$$
and therefore
$$
\bq_{(j)}^{\top} (\bB - z\bI)^{-1} \bq_{(j)} \approx \frac{\E v_n(z)}{1 + \tau_j \E v_n(z)}.
$$
Now, multiplying the left hand side by $\tau_j$ and averaging over $j$ gives
$$
\frac{1}{p}\sum_{j = 1}^{p} \tau_j \bq_{(j)}^{\top} (\bB - z\bI)^{-1} \bq_{(j)} = \frac{1}{p} \tr(\bB (\bB - z\bI)^{-1}) 
= \frac{1}{p} \tr(\bI_n + z (\bB - z\bI)^{-1} ) = \frac{1}{\gamma} + \frac{1}{\gamma} v_n(z)
\approx \frac{1}{\gamma} + \frac{1}{\gamma}\E v_n(z).
$$
Doing the same to the right hand side, we see that
\begin{equation}
\label{eqn:approximate-silverstein}
\frac{1}{\gamma} + \frac{1}{\gamma} \E v_n(z) \approx \frac{1}{\gamma} + \frac{1}{\gamma}\E v_n(z) \approx \frac{1}{p} \sum_{j = 1}^{p} \frac{\tau_j \E v_n(z)}{1 + \tau_j \E v_n(z)} \to \int \frac{\tau \E v_n(z)}{1 + \tau \E v_n(z)} \,dH(\tau).
\end{equation}
Rearranging~\eqref{eqn:approximate-silverstein} gives back~\eqref{eqn:silverstein}.

\section{Distribution of eigenvectors}
Define the following spectral distributions of a vector $\bbeta \in \Reals^p$:
$$
\wh{G}_n(\tau) := \frac{1}{\|\beta\|^2} \sum_{j = 1}^{p} (\bbeta_j^{\top} \bw_j)^2 \1\{\lambda_i(\bSigma) \leq \tau\}, \quad \wh{B}_n(\tau) := \frac{1}{\|\beta\|^2} \sum_{j = 1}^{p} (\bbeta_j^{\top} \bv_j)^2 \1\{\lambda_i(\bS) \leq \tau\}
$$ 
Obviously, taking $\bbeta = \bw$ means that $\wh{B}_n$ describes the correlation between the leading eigenspaces $\mathrm{span}\{\bv_1,\ldots,\bv_j\}$ and a true eigenvector of $\bSigma$.  The fundamental ideas used to derive the limit of the Stieltjes transform of $m_{\wh{B}_n}$ are the same as for eigenvalues, but the calculations seem a bit more tedious. 

\subsection{The fixed point equation}

\subsection{The high-level argument}
Introduce the resolvent notation 
$$
\bB = \frac{1}{n} \bSigma^{1/2} \bZ \bZ^{\top} \bSigma^{1/2}, \quad \bQ = (\bB - z\bI)^{-1}, \quad \bP = \bP(\hat{e}) = (x(\hat{e})\bSigma - z\bI)^{-1},
$$
where $\hat{e} = \frac{1}{p}\tr(\bSigma \bQ)$, and $x(\hat{e}) = \frac{1}{1 + \gamma \hat{e}}$. (Notice that the notation has shifted from last section.) The idea is to show that for a relatively general class of matrices $\bTheta$,
\begin{equation}
	\label{eqn:matrix-resolvent-convergence}
	\frac{1}{n}\tr(\bTheta \bQ- \bTheta \bP) \to 0.
\end{equation}
This is accomplished by leave-one-out arguments. By the resolvent identity,
$$
\bP - \bQ = \bP (\bQ^{-1} - \bP^{-1}) \bQ = \bP (\bB - x(\hat{e})\bSigma) \bQ. 
$$
Writing $\bB = \frac{1}{n}\sum_{i = 1}^{n} \bSigma^{1/2} \bz_i \bz_i^{\top} \bSigma^{1/2}$, we have from Sherman-Morrison that
\begin{align*}
\bB \bQ 
& = \frac{1}{n}\sum_{i = 1}^{n} \bSigma^{1/2} \bz_i \bz_i^{\top}  \bSigma^{1/2}  \bQ \\
& = \frac{1}{n}\sum_{i = 1}^{n} \bSigma^{1/2} \bz_i \bz_i^{\top}  \bSigma^{1/2}  \Big(\bQ_i - \frac{\bQ_i \bSigma^{1/2} \bz_i \bz_i^{\top}  \bSigma^{1/2} \bQ_i}{1 + \bz_i^{\top} \bSigma^{1/2} \bQ_i \bSigma^{1/2} \bz_i}\Big) \\
& = \frac{1}{n}\sum_{i = 1}^{n} \bSigma^{1/2} \bz_i  \Big(1 - \frac{\bz_i^{\top}  \bSigma^{1/2} \bQ_i \bSigma^{1/2} \bz_i }{1 + \bz_i^{\top} \bSigma^{1/2} \bQ_i \bSigma^{1/2} \bz_i}\Big) \bz_i^{\top}  \bSigma^{1/2} \bQ_i \\
& = \frac{1}{n}\sum_{i = 1}^{n} \frac{\bSigma^{1/2} \bz_i \bz_i^{\top}  \bSigma^{1/2} \bQ_i}{{1 + \bz_i^{\top} \bSigma^{1/2} \bQ_i \bSigma^{1/2} \bz_i}}.
\end{align*}
Now, assuming as before that differences between empirical means and expectations are negligible, so that we can move between them as needed, we have that
\begin{align*}
\tr(\wt{\bTheta} \bSigma^{1/2} \bz_i \bz_i^{\top}  \bSigma^{1/2} \bQ_i) & \approx \frac{1}{n}\tr(\wt{\bTheta} \bSigma \bQ) \\
\bz_i^{\top} \bSigma^{1/2} \bQ_i \bSigma^{1/2} \bz_i & \approx \frac{1}{n}\tr(\bSigma \bQ) = \gamma \hat{e}.
\end{align*}
Thus 
$$
\tr(\wt{\bTheta} \bB \bQ) \approx \frac{\tr(\wt{\bTheta} \bSigma \bQ)}{1 + \gamma \hat{e}} =  x(\hat{e}) \tr(\wt{\bTheta} \bSigma \bQ),
$$
establishing~\eqref{eqn:matrix-resolvent-convergence} upon taking $\wt{\bTheta} = \bTheta \bP$. 

One more empirical object $\hat{e}$ remains, but the preceding arguments show what its limit should be. In particular, taking $\bTheta = \bSigma$, we conclude that
$$
f(\hat{e}) := \frac{\hat{e}}{\gamma} - \frac{1}{n}\tr\big(\bSigma \bP(\hat{e})\big) \to 0.
$$
Letting $e$ be such that $f(e) = 0$, it can be shown that $f(\hat{e}) \to f(e) = 0$ implies $\hat{e} \to e$, and thus
\begin{equation}
	\frac{1}{n}\tr(\bTheta \bQ) - \frac{1}{n}\tr(\bTheta \bP(e)) \to 0.
\end{equation}


\section{Examples}

\subsection{Identity covariance}

The most basic case is $\bSigma = \bI, \gamma < 1$, $\hat{H}_n(\tau) = H_1(\tau) = \1\{\tau \leq 1\}$. In this case $\hat{F}_n(t)$ converges weakly to the {\bf Marchenko-Pastur distribution} $F_{\gamma}(t)$. The Marchenko-Pastur distribution has a density $f_{\gamma}(t)$ supported on $[t^{-},t^{+}] := [1 - \sqrt{\gamma}, 1 + \sqrt{\gamma}]$, with 
$$
f_{\gamma}(t) := \frac{\sqrt{(t^{+} - t)(t - t^{-})}}{2 \pi \gamma t}, \quad \textrm{for $t \in [t^{-},t^{+}]$.}
$$
If additionally $\gamma > 1$, then $F_{\gamma}$ has an atom at $0$ of magnitude $1 - 1/\gamma$. 

To show this, notice that when $H = \delta_1$ the self-consistency equation~\eqref{eqn:silverstein} reduces to
$$
z = -\frac{1}{v(z)} + \frac{\gamma}{1 + v(z)}.
$$
Using the definition of $m = \gamma v - (1 - \gamma)/z$, this can be rearranged to 
$$
m(z)= \frac{1}{(1 - \gamma) - \gamma m z - z}.
$$
Taking the resulting quadratic equation and solving for $m$ yields two solutions, 
$$
\frac{(1 - \gamma - z) \pm \sqrt{(1 + \gamma - z)^2 - 4 \gamma}}{2 \gamma z}.
$$
If we adopt the usual convention that $\sqrt{z} \in \mathbb{C}_{+}$ refers to the positive branch of a complex number, then recalling that $m(z) \in \mathbb{C}_{+}$, it follows that 
\begin{align*}
m_{\gamma}(z) 
& = \frac{(1 - \gamma - z) + \sqrt{(1 + \gamma - z)^2 - 4 \gamma}}{2 \gamma z} \\
& = \frac{(1 - \gamma - z) + \sqrt{( (1 + \sqrt{\gamma})^2 - z)((1 - \sqrt{\gamma})^2 - z)}}{2 \gamma z}.
\end{align*}
From here we can compute the density of the Marchenko-Pastur distribution by applying the Stieltjes inversion formula:
\begin{align*}
f_{\gamma}(t) 
& = \frac{1}{\pi}\lim_{\varepsilon \to 0} \Im m_{\gamma}(t + i \varepsilon) \\
& = \frac{1}{\pi} \Im \frac{\sqrt{( (1 + \sqrt{\gamma})^2 - t)((1 - \sqrt{\gamma})^2 - t)}}{2 \gamma t} \\
& = \frac{1}{\pi} \frac{\sqrt{( (1 + \sqrt{\gamma})^2 - t)(t - (1 - \sqrt{\gamma})^2)}}{2 \gamma t}.
\end{align*}

	
	
\end{document}