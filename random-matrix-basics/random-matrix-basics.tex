\documentclass{article}

% Packages
\usepackage[utf8]{inputenc} % allow utf-8 input
\usepackage[T1]{fontenc}    % use 8-bit T1 fonts
\usepackage{booktabs}       % professional-quality tables
\usepackage{nicefrac}       % compact symbols for 1/2, etc.
\usepackage{microtype}      % microtypography
\usepackage{times}             % times font
\usepackage{mathrsfs}      % Added by Alden, for script font.
\usepackage{ragged2e}     % Added by Alden, for no indent.
\usepackage{parskip}        % Added by Alden, for skips between paragraphs.

\usepackage[round]{natbib}
\usepackage{amssymb,amsmath,amsthm,bbm}
\usepackage[margin=1in]{geometry}
\usepackage{verbatim,float,url,dsfont}
\usepackage{graphicx,subcaption,psfrag} % Alden changed subfigure to subcaption
\usepackage{algorithm,algorithmic}
\usepackage{mathtools}
\usepackage[shortlabels]{enumitem}       % Alden added shortlabels option
\usepackage[colorlinks=true,citecolor=blue,urlcolor=blue,linkcolor=blue]{hyperref}
\usepackage{multirow}

% Theorems and such
\newtheorem{theorem}{Theorem}
\newtheorem{lemma}{Lemma}
\newtheorem{corollary}{Corollary}
\newtheorem{proposition}{Proposition}
\theoremstyle{definition}
\newtheorem{remark}{Remark}
\newtheorem{definition}{Definition}
\newtheorem{example}{Example} % Added by Alden
\newtheorem{conjecture}{Conjecture} % Added by Alden

% Assumption
\newtheorem*{assumption*}{\assumptionnumber}
\providecommand{\assumptionnumber}{}
\makeatletter
\newenvironment{assumption}[2]{
  \renewcommand{\assumptionnumber}{Assumption #1#2}
  \begin{assumption*}
  \protected@edef\@currentlabel{#1#2}}
{\end{assumption*}}
\makeatother

% Widebar
\makeatletter
\newcommand*\rel@kern[1]{\kern#1\dimexpr\macc@kerna}
\newcommand*\widebar[1]{%
  \begingroup
  \def\mathaccent##1##2{%
    \rel@kern{0.8}%
    \overline{\rel@kern{-0.8}\macc@nucleus\rel@kern{0.2}}%
    \rel@kern{-0.2}%
  }%
  \macc@depth\@ne
  \let\math@bgroup\@empty \let\math@egroup\macc@set@skewchar
  \mathsurround\z@ \frozen@everymath{\mathgroup\macc@group\relax}%
  \macc@set@skewchar\relax
  \let\mathaccentV\macc@nested@a
  \macc@nested@a\relax111{#1}%
  \endgroup
}
\makeatother

% Min and max
\newcommand{\argmin}{\mathop{\mathrm{argmin}}}
\newcommand{\argmax}{\mathop{\mathrm{argmax}}}
\newcommand{\minimize}{\mathop{\mathrm{minimize}}}
\newcommand{\st}{\mathop{\mathrm{subject\,\,to}}}
\DeclareMathOperator*{\esssup}{ess\,sup} % Added by Alden

% Shortcuts
\def\R{\mathbb{R}}
\def\C{\mathbb{C}}
\def\E{\mathbb{E}}
\def\P{\mathbb{P}}
\def\T{\mathsf{T}}
\def\Cov{\mathrm{Cov}}
\def\Var{\mathrm{Var}}
\def\half{\frac{1}{2}}
\def\tr{\mathrm{tr}}
\def\df{\mathrm{df}}
\def\dim{\mathrm{dim}}
\def\col{\mathrm{col}}
\def\row{\mathrm{row}}
\def\nul{\mathrm{null}}
\def\rank{\mathrm{rank}}
\def\nuli{\mathrm{nullity}}
\def\spa{\mathrm{span}}
\def\sign{\mathrm{sign}}
\def\supp{\mathrm{supp}}
\def\diag{\mathrm{diag}}
\def\aff{\mathrm{aff}}
\def\conv{\mathrm{conv}}
\def\dom{\mathrm{dom}}
\def\hy{\hat{y}}
\def\hf{\hat{f}}
\def\hmu{\hat{\mu}}
\def\halpha{\hat{\alpha}}
\def\hbeta{\hat{\beta}}
\def\htheta{\hat{\theta}}
\def\cA{\mathcal{A}}
\def\cB{\mathcal{B}}
\def\cD{\mathcal{D}}
\def\cE{\mathcal{E}}
\def\cF{\mathcal{F}}
\def\cG{\mathcal{G}}
\def\cK{\mathcal{K}}
\def\cH{\mathcal{H}}
\def\cI{\mathcal{I}}
\def\cL{\mathcal{L}}
\def\cM{\mathcal{M}}
\def\cN{\mathcal{N}}
\def\cP{\mathcal{P}}
\def\cS{\mathcal{S}}
\def\cT{\mathcal{T}}
\def\cW{\mathcal{W}}
\def\cX{\mathcal{X}}
\def\cY{\mathcal{Y}}
\def\cZ{\mathcal{Z}}

%%% Begin Alden's additions
\newcommand{\Ebb}{\mathbb{E}}
\newcommand{\Pbb}{\mathbb{P}}
\newcommand{\dotp}[2]{\langle #1, #2 \rangle}
\newcommand{\wt}[1]{\widetilde{#1}}
\newcommand{\wh}[1]{\widehat{#1}}
\newcommand{\mc}[1]{\mathcal{#1}}
\newcommand{\Reals}{\mathbb{R}} % Same thing as Ryan's \R
\newcommand{\Rd}{\Reals^d}
\newcommand{\wb}[1]{\widebar{#1}}
\newcommand{\floor}[1]{\left\lfloor #1 \right\rfloor}
\newcommand{\ceil}[1]{\left\lceil #1 \right\rceil}
\newcommand{\1}{\mathbf{1}}
\newcommand{\bj}{{\bf j}}
\newcommand{\restr}[2]{\ensuremath{\left.#1\right|_{#2}}}
\newcommand{\TV}{\mathrm{TV}}

\DeclareFontFamily{U}{mathx}{\hyphenchar\font45}
\DeclareFontShape{U}{mathx}{m}{n}{<-> mathx10}{}
\DeclareSymbolFont{mathx}{U}{mathx}{m}{n}
\DeclareMathAccent{\wc}{0}{mathx}{"71}


%%% End Alden's Additions
\usepackage{lmodern}
\usepackage{xcolor}
\usepackage{diagbox}
\usepackage{xr-hyper}
\graphicspath{ {../../code/fig/} }

\newcommand{\bx}{\boldsymbol{x}}
\newcommand{\bv}{\boldsymbol{v}}
\newcommand{\bX}{\boldsymbol{X}}
\newcommand{\bSigma}{\boldsymbol{\Sigma}}
\newcommand{\bS}{\boldsymbol{S}}
\newcommand{\bI}{\boldsymbol{I}}

\title{ {\bf The Basics of Random Matrix Theory} }

\begin{document}
	
\maketitle
\RaggedRight

\abstract{This document records some elementary background on random matrix theory for sample covariance matrices.}

\section{Introduction}

Suppose random vectors $\bx_1,\ldots,\bx_n \in \Reals^p$ are independently sampled from a Normal distribution with mean $0$ and covariance $\bSigma$. The sample covariance matrix is
$$
\bS = \frac{1}{n}\sum_{i = 1}^{n} \bx_i \bx_i^{\top} = \frac{1}{n}\bX \bX^{\top},
$$ 
where $\bX \in \Reals^{n \times p}$ is matrix with rows $\bX_{i\cdot} = \bx_i$. Random matrix theory describes the asymptotic behavior of the sample eigenvalues $\lambda_i(\bS) = \lambda_i(\bS)$ and eigenvectors $\bv_i = \bv_i(\bS)$ of $\bS$, in the \emph{proportional asymptotics} limit $n,p \to \infty, p/n \to \gamma \in (0,\infty)$. 

\subsection{Stieltjes transform}
The Stieltjes transform is a key analytical device used to prove asymptotic convergence of spectral distributions of random matrices. The {\bf Stieltjes transform} of a measure $\mu$ is defined for $z \in \mathbb{C}\setminus\supp(\mu)$ as
$$
m_{\mu}(z) := \int \frac{1}{t - z} \,d\mu(t).
$$
The measure $\mu$ can be recovered from its Stieltjes transform via the {\bf Stieltjes inversion formula}. At a given $\tau \in \supp(\mu)$, if $\lim_{\varepsilon \to 0} \Im m_{\mu}(\tau + \varepsilon i)$ exists then $m_{\mu}$ is continuous at $t$, with density $f_{\mu}(\tau) = \frac{1}{\pi}\lim_{\varepsilon} \Im m_{\mu}(\tau + \varepsilon i)$. To see this:
\begin{align*}
	\lim_{\varepsilon \to 0}\Im m_{\mu}(\tau + \varepsilon i) 
	& = 
	\lim_{\varepsilon \to 0}\int \Im(\frac{1}{t - (\tau + \varepsilon i)} \,d\mu(t) \\
	& = \lim_{\varepsilon \to 0} \int \frac{\varepsilon}{(t - \tau)^2 + \varepsilon^2} \,d\mu(t) \\
	& = \frac{1}{\pi}f(\tau).
\end{align*} 

\section{Marchenko-Pastur distribution}
Let $\hat{H}_n(\tau)$ represent the empirical spectral distribution of the eigenvalues of $\bSigma$:
$$
\hat{H}_n(t) := \frac{1}{p} \sum_{i = 1}^{p} \1\{\lambda_i(\bSigma) \leq t\}.
$$
Suppose that $\hat{H}_n(t)$ converges weakly to a limiting spectral distribution $H(t)$ as $p \to \infty$. Then the spectral distribution $\hat{F}_n(t)$ of the sample covariance $\bS$ also converges to continuum limit $F_{\gamma,H}(t)$, which we will call a {\bf generalized Marchenko-Pastur distribution}. Random matrix theory provides the analytical tools necessary to prove this result and describe the resulting continuum limit.

The most basic case is $\bSigma = \bI, \gamma < 1$, $\hat{H}_n(\tau) = H_1(\tau) = \1\{\tau \leq 1\}$. In this case $\hat{F}_n(t)$ converges weakly to the {\bf Marchenko-Pastur distribution} $F_{\gamma}(t)$. The Marchenko-Pastur distribution has a density $f_{\gamma}(t)$ supported on $[t^{-},t^{+}] := [1 - \sqrt{\gamma}, 1 + \sqrt{\gamma}]$, with 
$$
f_{\gamma}(t) := \frac{\sqrt{(t^{+} - t)(t - t^{-})}}{2 \pi \gamma t}, \quad \textrm{for $t \in [t^{-},t^{+}]$.}
$$
If additionally $\gamma > 1$, then $F_{\gamma}$ has an atom at $0$ of magnitude $1 - 1/\gamma$. 

\subsection{Silverstein's equation}
Let $m_{\gamma}$ be the Stieltjes transform of the limiting generalized Marchenko-Pastur distribution $F_{\gamma,H}$. It can be shown that $m_{\gamma}$ satisfies the fixed point equation
$$
z = -\frac{1}{m(z)} + \gamma \int \frac{t}{1 + tm(z)} \,dH(t), \quad \textrm{for all $z \in \C_{+}$.}
$$
In the specific case of $\bSigma = \bI$ and $H = H_1$, this is simply
$$
z = -\frac{1}{m(z)} + \gamma \frac{1}{1 + m(z)}.
$$
Rearranging and solving the resulting quadratic equation for $m$ gives
$$
m(z) = 
$$

\subsection{Stieltjes inversion formula}

	
	
\end{document}