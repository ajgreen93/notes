\documentclass{article}

% Packages
\usepackage[utf8]{inputenc} % allow utf-8 input
\usepackage[T1]{fontenc}    % use 8-bit T1 fonts
\usepackage{booktabs}       % professional-quality tables
\usepackage{nicefrac}       % compact symbols for 1/2, etc.
\usepackage{microtype}      % microtypography
\usepackage{times}             % times font
\usepackage{mathrsfs}      % Added by Alden, for script font.
\usepackage{ragged2e}     % Added by Alden, for no indent.
\usepackage{parskip}        % Added by Alden, for skips between paragraphs.

\usepackage[round]{natbib}
\usepackage{amssymb,amsmath,amsthm,bbm}
\usepackage[margin=1in]{geometry}
\usepackage{verbatim,float,url,dsfont}
\usepackage{graphicx,subcaption,psfrag} % Alden changed subfigure to subcaption
\usepackage{algorithm,algorithmic}
\usepackage{mathtools}
\usepackage[shortlabels]{enumitem}       % Alden added shortlabels option
\usepackage[colorlinks=true,citecolor=blue,urlcolor=blue,linkcolor=blue]{hyperref}
\usepackage{multirow}

% Theorems and such
\newtheorem{theorem}{Theorem}
\newtheorem{lemma}{Lemma}
\newtheorem{corollary}{Corollary}
\newtheorem{proposition}{Proposition}
\theoremstyle{definition}
\newtheorem{remark}{Remark}
\newtheorem{definition}{Definition}
\newtheorem{example}{Example} % Added by Alden
\newtheorem{conjecture}{Conjecture} % Added by Alden

% Assumption
\newtheorem*{assumption*}{\assumptionnumber}
\providecommand{\assumptionnumber}{}
\makeatletter
\newenvironment{assumption}[2]{
  \renewcommand{\assumptionnumber}{Assumption #1#2}
  \begin{assumption*}
  \protected@edef\@currentlabel{#1#2}}
{\end{assumption*}}
\makeatother

% Widebar
\makeatletter
\newcommand*\rel@kern[1]{\kern#1\dimexpr\macc@kerna}
\newcommand*\widebar[1]{%
  \begingroup
  \def\mathaccent##1##2{%
    \rel@kern{0.8}%
    \overline{\rel@kern{-0.8}\macc@nucleus\rel@kern{0.2}}%
    \rel@kern{-0.2}%
  }%
  \macc@depth\@ne
  \let\math@bgroup\@empty \let\math@egroup\macc@set@skewchar
  \mathsurround\z@ \frozen@everymath{\mathgroup\macc@group\relax}%
  \macc@set@skewchar\relax
  \let\mathaccentV\macc@nested@a
  \macc@nested@a\relax111{#1}%
  \endgroup
}
\makeatother

% Min and max
\newcommand{\argmin}{\mathop{\mathrm{argmin}}}
\newcommand{\argmax}{\mathop{\mathrm{argmax}}}
\newcommand{\minimize}{\mathop{\mathrm{minimize}}}
\newcommand{\st}{\mathop{\mathrm{subject\,\,to}}}
\DeclareMathOperator*{\esssup}{ess\,sup} % Added by Alden

% Shortcuts
\def\R{\mathbb{R}}
\def\C{\mathbb{C}}
\def\E{\mathbb{E}}
\def\P{\mathbb{P}}
\def\T{\mathsf{T}}
\def\Cov{\mathrm{Cov}}
\def\Var{\mathrm{Var}}
\def\half{\frac{1}{2}}
\def\tr{\mathrm{tr}}
\def\df{\mathrm{df}}
\def\dim{\mathrm{dim}}
\def\col{\mathrm{col}}
\def\row{\mathrm{row}}
\def\nul{\mathrm{null}}
\def\rank{\mathrm{rank}}
\def\nuli{\mathrm{nullity}}
\def\spa{\mathrm{span}}
\def\sign{\mathrm{sign}}
\def\supp{\mathrm{supp}}
\def\diag{\mathrm{diag}}
\def\aff{\mathrm{aff}}
\def\conv{\mathrm{conv}}
\def\dom{\mathrm{dom}}
\def\hy{\hat{y}}
\def\hf{\hat{f}}
\def\hmu{\hat{\mu}}
\def\halpha{\hat{\alpha}}
\def\hbeta{\hat{\beta}}
\def\htheta{\hat{\theta}}
\def\cA{\mathcal{A}}
\def\cB{\mathcal{B}}
\def\cD{\mathcal{D}}
\def\cE{\mathcal{E}}
\def\cF{\mathcal{F}}
\def\cG{\mathcal{G}}
\def\cK{\mathcal{K}}
\def\cH{\mathcal{H}}
\def\cI{\mathcal{I}}
\def\cL{\mathcal{L}}
\def\cM{\mathcal{M}}
\def\cN{\mathcal{N}}
\def\cP{\mathcal{P}}
\def\cS{\mathcal{S}}
\def\cT{\mathcal{T}}
\def\cW{\mathcal{W}}
\def\cX{\mathcal{X}}
\def\cY{\mathcal{Y}}
\def\cZ{\mathcal{Z}}

%%% Begin Alden's additions
\newcommand{\Ebb}{\mathbb{E}}
\newcommand{\Pbb}{\mathbb{P}}
\newcommand{\dotp}[2]{\langle #1, #2 \rangle}
\newcommand{\wt}[1]{\widetilde{#1}}
\newcommand{\wh}[1]{\widehat{#1}}
\newcommand{\mc}[1]{\mathcal{#1}}
\newcommand{\Reals}{\mathbb{R}} % Same thing as Ryan's \R
\newcommand{\Rd}{\Reals^d}
\newcommand{\wb}[1]{\widebar{#1}}
\newcommand{\floor}[1]{\left\lfloor #1 \right\rfloor}
\newcommand{\ceil}[1]{\left\lceil #1 \right\rceil}
\newcommand{\1}{\mathbf{1}}
\newcommand{\bj}{{\bf j}}
\newcommand{\restr}[2]{\ensuremath{\left.#1\right|_{#2}}}
\newcommand{\TV}{\mathrm{TV}}

\DeclareFontFamily{U}{mathx}{\hyphenchar\font45}
\DeclareFontShape{U}{mathx}{m}{n}{<-> mathx10}{}
\DeclareSymbolFont{mathx}{U}{mathx}{m}{n}
\DeclareMathAccent{\wc}{0}{mathx}{"71}


%%% End Alden's Additions
\usepackage{lmodern}
\usepackage{xcolor}
\usepackage{diagbox}
\usepackage{xr-hyper}

\newcommand{\bx}{\boldsymbol{x}}
\newcommand{\bz}{\boldsymbol{z}}
\newcommand{\bv}{\boldsymbol{v}}
\newcommand{\bw}{\boldsymbol{w}}
\newcommand{\bg}{\boldsymbol{g}}
\newcommand{\bq}{\boldsymbol{q}}
\newcommand{\bX}{\boldsymbol{X}}
\newcommand{\bbeta}{\boldsymbol{\beta}}
\newcommand{\bSigma}{\boldsymbol{\Sigma}}
\newcommand{\bTheta}{\boldsymbol{\Theta}}
\newcommand{\bS}{\boldsymbol{S}}
\newcommand{\bT}{\boldsymbol{T}}
\newcommand{\bB}{\boldsymbol{B}}
\newcommand{\bG}{\boldsymbol{G}}
\newcommand{\bQ}{\boldsymbol{Q}}
\newcommand{\bP}{\boldsymbol{P}}
\newcommand{\bR}{\boldsymbol{R}}
\newcommand{\bI}{\boldsymbol{I}}
\newcommand{\bZ}{\boldsymbol{Z}}
\newcommand{\Complex}{\mathbb{C}}

\newcommand{\interior}{\mathrm{int}}
\newcommand{\range}{\mathrm{range}}

\newcommand{\convweak}{\Rightarrow}

\title{ {\bf The Basics of Random Matrix Theory} }

\begin{document}
	
\maketitle
\RaggedRight

\begin{abstract}
	This document records some elementary background on random matrix theory for sample covariance matrices.
\end{abstract}

\section{Introduction}

Suppose random vectors $\bx_1,\ldots,\bx_n \in \Reals^p$ are independently sampled from a Normal distribution with mean $0$ and covariance $\bSigma$. The sample covariance matrix is
$$
\bS = \frac{1}{n}\sum_{i = 1}^{n} \bx_i \bx_i^{\top} = \frac{1}{n}\bX \bX^{\top},
$$ 
where $\bX \in \Reals^{n \times p}$ is matrix with rows $\bX_{i\cdot} = \bx_i$. Random matrix theory describes the asymptotic behavior of the sample eigenvalues $\lambda_i(\bS) = \lambda_i(\bS)$ and eigenvectors $\bv_i = \bv_i(\bS)$ of $\bS$, in the \emph{proportional asymptotics} limit $n,p \to \infty, p/n \to \gamma \in (0,\infty)$. This is achieved by through the device of the Stieltjes transform which packages the spectral information of $\bS$, and is the key analytical tool used to prove asymptotic convergence of spectral distributions of random matrices.

This note analyzes the asymptotic behavior of the Stieltjes transform of $\bS$, along with associated other measures. Throughout, we will assume that the population covariance is diagonal $\bSigma = \bT = \diag(\{\tau_j\})$, to simplify the proofs. When the data are Gaussian, as assumed here, this is without loss of generality.  

\subsection{Stieltjes transform}
The {\bf Stieltjes transform} of a measure $\mu$ is defined for $z \in \mathbb{C}\setminus\supp(\mu)$ as
$$
m_{\mu}(z) := \int \frac{1}{t - z} \,d\mu(t), \quad m_{\mu}(z) \in C_{+}.
$$
It is the case that $m_{\mu}$ is continuous in $\Complex_{+}$, though it is not continuous over all of its support. We will extend $m_{\mu}(z)$ by continuity to be defined for all $z \in \mathbb{C} \setminus0$, that is, for $x \in \supp(\mu) \setminus 0$, we will define $m_{\mu}(x) := \lim_{\varepsilon \to 0} m_{\mu}(x + \varepsilon i)$. 
The measure $\mu$ can be recovered from its Stieltjes transform via the {\bf Stieltjes inversion formula}. At a given $\tau \in \supp(\mu)$, if $\lim_{\varepsilon \to 0} \Im m_{\mu}(\tau + \varepsilon i)$ exists then $m_{\mu}$ is continuous at $t$, with density $f_{\mu}(\tau) = \frac{1}{\pi}\lim_{\varepsilon} \Im m_{\mu}(\tau + \varepsilon i)$. To see this:
\begin{align*}
	\lim_{\varepsilon \to 0}\Im m_{\mu}(\tau + \varepsilon i) 
	& = 
	\lim_{\varepsilon \to 0}\int \Im\Big(\frac{1}{t - (\tau + \varepsilon i)}\Big) \,d\mu(t) \\
	& = \lim_{\varepsilon \to 0} \int \frac{\varepsilon}{(t - \tau)^2 + \varepsilon^2} \,d\mu(t) \\
	& = \pi f(\tau),
\end{align*} 
with the last equality following since the Poisson kernels $P_{\varepsilon}(x) = \frac{1}{\pi} \frac{\varepsilon}{x^2 + \varepsilon^2}$ form a family of approximations to the identity. On the other hand, it is also the case that
$$
\lim_{\varepsilon \to 0} - \varepsilon i m_{\mu}(t_0 + \varepsilon i) 
= \lim_{\varepsilon \to 0} - \varepsilon i \int \frac{1}{t - (t_0 + \varepsilon i)} \,d\mu(t)
= \lim_{\varepsilon \to 0} \mu(t_0) + o_{\varepsilon}(1)
= \mu(t_0),
$$  
which can be used to recover the atoms of $\mu$.

\subsection{Companion transform}
The {\bf companion} to the Stieltjes transform is defined for $z \in \mathbb{C} \setminus (\supp(\mu) \cup \{0\})$ as,
$$
v_{\mu}(z) := \gamma m_{\mu}(z) - \frac{1 - \gamma}{z}.
$$
This is simply the Stieltjes transform of $\underline{\mu} = \gamma \mu + (1 - \gamma) \delta_{0}$, and it describes the limiting spectrum of the matrix
$$
\frac{1}{p} \sum_{i = 1}^{p} \bx_{(j)} \bx_{(j)}^{\top}.
$$

\subsection{Inverse of the companion Stieltjes transform}
\label{sec:inverse-companion-transform}

Finally, the inverse of the companion Stieltjes transform $v^{-1}$ satisfies
$$
v_{\mu}^{-1}(v_{\mu}(z)) = z, \quad \textrm{for all}\quad z \in \mathbb{C}\setminus\supp(\mu).
$$
The inverse can be useful in computing the support of $\mu$. It can be shown that a point $x \in \Reals$ does not belong to the interior of the support of $\mu$, $x \not\in \mathrm{int}~\supp(\mu)$ if and only if $v_{\mu}(x) \in \Reals$; this is intuitively correct as $v_{\mu}(x) \in \Reals$ if and only if $f_{\mu}(x) = 0$. It follows that 
$$
\Reals \setminus \mathrm{int}~\supp(\mu) = v_{\mu}^{-1}\Big(\mathrm{range}(v_{\mu}(x)) \cap \Reals\Big).
$$
If further follows that if $x \in \mathrm{range}(v_{\mu}(x))$ then $v_{\mu}'(x) > 0$, therefore at $m = v_{\mu}(x)$ we have $(v_{\mu}^{-1})'(m) > 0$. In other words, 
$$
\mathrm{range}(v_{\mu}(x)) \cap \Reals \subseteq \{m \in \Reals: m \neq 0, (v_{\mu}^{-1})'(m) > 0 \} := \mc{D}_{\mu}.
$$
In fact, the two are exactly equal, and it follows that $v_{\mu}^{-1}(\mc{D}_{\mu}) = \Reals \setminus \mathrm{int}~\supp(\mu)$.

\section{Distribution of eigenvalues}
Let $\hat{H}_n(\tau)$ represent the empirical spectral distribution of the eigenvalues of $\bSigma$:
$$
\hat{H}_n(t) := \frac{1}{p} \sum_{i = 1}^{p} \1\{\lambda_i(\bSigma) \leq t\}.
$$
Suppose that $\hat{H}_n(t)$ converges weakly to a limiting spectral distribution $H(t)$ as $p \to \infty$. Then the spectral distribution $\hat{F}_n(t)$ of the sample covariance $\bS$ also converges to continuum limit $F_{\gamma,H}(t)$, which we will call a {\bf generalized Marchenko-Pastur distribution}. Random matrix theory provides the analytical tools necessary to prove this result and describe the resulting continuum limit. We give the result and describe some of the main ideas of the proof.

\subsection{The fixed point equation}
Let $m_{\gamma,H}$ be the Stieltjes transform of the limiting generalized Marchenko-Pastur distribution $F_{\gamma,H}$. The companion transform is $v_{\gamma,H} := \gamma m_{\gamma,H}(z) - \frac{(1 - \gamma)}{z}$. It can be shown, via the leave-one-out approach, that the companion transform satisfies the self-consistency equation
\begin{equation}
	\label{eqn:silverstein}
	z = -\frac{1}{v(z)} + \gamma \int \frac{\tau}{1 + \tau v(z)} \,dH(\tau), \quad \textrm{for all $z \in \C_{+}$.}
\end{equation}
Equation~\eqref{eqn:silverstein} can be manipulated to show that the inverse of the companion transform satisfies
\begin{equation}
\label{eqn:inverse-fixed-point}
v_{\gamma,H}^{-1}(z) = -\frac{1}{z} + \gamma\int\frac{\tau}{1 + \tau z} \,dH(\tau). 
\end{equation} 
This will be useful in computing the support of $F_{\gamma,H}$. 

\subsection{Atoms}

Using~\eqref{eqn:silverstein}, we can describe the behavior of $v(\eta i)$ as $\eta \to 0$. First, we rearrange~\eqref{eqn:silverstein} to
\begin{equation}
	\label{eqn:silverstein-2}
	zv(z) = \gamma - 1 - \gamma \int \frac{1}{1 + \tau v(z)} \,dH(\tau).
\end{equation}
Notice that if $\gamma < 1$ then the function $\tilde{v}(z) = (\gamma - 1)/z$ has range $\Complex_{+}$, and additionally, 
$$
\lim_{\eta \to 0} \eta i \tilde{v}(\eta i) = \gamma - 1 = \lim_{\eta \to 0} \gamma - 1 - \gamma \int \frac{1}{1 + \tau/(\eta i)} \,dH(\tau).
$$
It follows that $\lim_{\eta \to 0} v(\eta i) = \lim_{\eta \to 0} \tilde{v}(\eta i) = \infty$. On the other hand, if $\gamma > 1$ then we suppose $\lim_{\eta \to 0} v(\eta i) = v_0 \in \Reals$. In that case we have from~\eqref{eqn:silverstein-2} that
\begin{equation}
	\label{eqn:stieltjes-at-zero}
	\frac{\gamma - 1}{\gamma} =  \int \frac{1}{1 + \tau v_0} \,dH(\tau),
\end{equation}
which uniquely defines $v_0 > 0$.


\subsection{Proof of~\eqref{eqn:silverstein}}
\label{subsec:proof-of-eqn:silverstein}

The self-consistency equation~\eqref{eqn:silverstein} is proven by leave-one-out arguments. What follows is intended to highlight the key parts of the argument; proofs of concentration and verification of other necessary regularity conditions are omitted. 

Begin by introducing the notation
$$
\bR(z) := [\bS - z \bI]^{-1}, m_n(z) := \frac{1}{p} \tr(\bR(z)),
$$
for the resolvent and Stieltjes transform of $\bS$, respectively. The following arguments will show that the Stieltjes transform converges as $n, p \to \infty, n/p \to \gamma \in (0,\infty)$ to a deterministic function defined in terms of a fixed point equation. Notice that by rotational invariance of the trace, we may as well assume that $\bSigma = \bT$ is a diagonal matrix, and we will do so throughout. 

The diagonal elements of the resolvent can be computed by block matrix inversion. We have that 
\begin{equation}
	\label{eqn:leave-one-out-resolvent}
	\bR(z)_{11} = \bigg(\bS_{11} - z - \bS_{1\cdot}^{\top} \bR(z)_{-1} \bS_{1\cdot}\bigg)^{-1},
\end{equation}
where $\bS_{-1} \in \Reals^{(p - 1) \times (p - 1)}$, and $\bS_{1\cdot}$ is the $1$st row of $\bS$; these can also be written as 
$$
\bS = \frac{1}{n} \bT^{1/2} \bG^{\top} \bG \bT^{1/2}, \quad \bS_{11} = \frac{\tau_1}{n}\|\bg_1\|^2, \quad \bS_{1\cdot} = \frac{\sqrt{\tau_1}}{n} \bg_1^{\top} \bG \bT^{1/2}.
$$
Plugging back into~\eqref{eqn:leave-one-out-resolvent}, we have
\begin{equation}
	\label{eqn:leave-one-out-resolvent-2}
	\bR(z)_{11} = \bigg(\frac{\tau_1}{n}\|\bg_1\|^2 - z - \frac{\tau_1}{n^2} \bg_1^{\top} \bX_{-1}^{\top} \bR(z)_{-1} \bX_{-1}^{\top} \bg_1\bigg)^{-1}.
\end{equation}
Each of the two terms in~\eqref{eqn:leave-one-out-resolvent-2} concentrates around its expectation: as $n,p \to \infty$,
\begin{equation}
	\label{eqn:pf-eqn:silverstein-1}
	\frac{\tau_1}{n}\|\bg_1\|^2 \sim \frac{\tau_1}{n}\mathbb{E}[\|\bg_1\|^2] \to \tau_1,
\end{equation}
while almost surely,
\begin{align}
	\frac{\tau_1}{n^2} \bg_1^{\top} \bX_{-1}^{\top} \bR(z)_{-1} \bX_{-1}^{\top} \bg_1 
	& \sim \frac{\tau_1}{n^2} \mathbb{E}_{\bX_{-1}}\Big[\bg_1^{\top} \bX_{-1}^{\top} \bR(z)_{-1} \bX_{-1}^{\top} \bg_1\Big] \\
	& = \frac{\tau_1}{n^2} \tr\Big(\bX_{-1}^{\top} \bR(z)_{-1} \bX_{-1}^{\top}\Big) \\
	& = \frac{\tau_1}{n^2} \tr\Big(\bR(z)_{-1} \bX_{-1}^{\top} \bX_{-1} \Big) \\
	& = \frac{\tau_1}{n} \tr\Big(\bR(z)_{-1} \bS_{-1} \Big) \\
	& = \frac{\tau_1}{n} \tr\Big(\bI_{p - 1} + z \bR(z)_{-1} \Big) \\
	& = \tau_1 \gamma + \tau_1 z \gamma m_{n,-1}(z) \\
	& \sim \tau_1 \gamma + \tau_1 z \gamma m_{n}(z).
	\label{eqn:almost-sure-convergence-diagonal-resolvent}
\end{align}
Here the notation $A_n \sim B_n$ means $A_n/B_n \to 1$ almost surely as $n \to \infty$. Equation~\eqref{eqn:pf-eqn:silverstein-1} and the first line of the above string of equalities follow by concentration inequalities (e.g. Hanson-Wright inequality) and suitable preliminary results implying that the Frobenius norm of $\bR$ is $\mc{O}(p)$.  The last line of the above string of equalities follows since the first diagonal entry of $\bR(z)_{-1}$ is $\mc{O}(1)$. Therefore,
\begin{equation}
	\label{eqn:resolvent-diagonal}
	\bR(z)_{11} \sim \bigg(\tau_1 - z - \tau_1 \gamma(1 + z m_n(z))\bigg)^{-1}.
\end{equation}

Repeating this calculation for $\bR(z)_{jj}, j = 1,\ldots,p$, and averaging over $j$, we obtain
\begin{equation*}
	m_n(z) \sim \frac{1}{p} \sum_{j = 1}^{p} \frac{1}{\tau_j - z - \tau_j( \gamma + z \gamma m_n(z))} \sim \int \frac{1}{\tau(1 - \gamma) - z - \tau z \gamma m_n(z)} \,dH(\tau).
\end{equation*}
It follows that the limit $m(z) = \lim_{n \to \infty} m_n(z)$ satisfies
\begin{align}
	m(z) 
	& = \int \frac{1}{\tau(1 - \gamma) - z - \tau z \gamma m(z)} \,dH(\tau) \\ 
	& = -\frac{1}{z} \int \frac{1}{1 + \tau v(z)} \,dH(\tau).
\end{align}
Recalling that $\frac{m(z)}{\gamma} = v(z) + \frac{(1 - \gamma)}{z}$, some algebra shows that
\begin{align*}
	-v(z) 
	& = -\frac{(1 - \gamma)}{z} -\frac{\gamma}{z} \int \frac{1}{1 + \tau v(z)} \,dH(\tau) \\
	& = -\frac{1}{z} + \frac{\gamma}{z} \int \frac{\tau v(z)}{1 + \tau v(z)} \,dH(\tau).
\end{align*}
Rearranging terms then yields the claimed result~\eqref{eqn:silverstein}.


\section{Spectral measure of population covariance}
\label{sec:spectral-measure-population-covariance}
Let $\bv_i$ represent the $i$th eigenvector of $\bS$. Consider the weighted spectral measure:
$$
\wh{Q}_n(\theta) = \frac{1}{p} \sum_{i = 1}^{p} \dotp{\bv_i}{\bSigma \bv_i} \cdot \1\{\lambda_i(\bS) \leq \theta\}.
$$
This can be thought of as the empirical spectral measure of the matrix $\bSigma$.
We show how to build upon the results of the preceding section to establish the weak convergence of $\wh{Q}_n$ as $n \to \infty, p \to \infty, p/n \to \gamma \in (0,\infty)$. 

As usual, this will be accomplished by establishing convergence of the Stieltjes transform, 
\begin{equation}
	\label{eqn:esd-of-covariance}
	m_{\wh{Q}_n}(z) = \frac{1}{p}\tr(\bSigma \bR(z)), \quad \bR = [\bS - z \bI]^{-1},
\end{equation}
and then applying the inversion formula.

\subsection{Convergence of the Stieltjes transform~\eqref{eqn:esd-of-covariance}}
We will assume without loss of generality that $\bSigma = \bT$ is diagonal, so
$$
m_{\wh{Q}_n}(z) = \frac{1}{p} \sum_{i = 1}^{p} \tau_j \bR(z)_{jj}.
$$
Equation \eqref{eqn:almost-sure-convergence-diagonal-resolvent} in the previous section shows that the diagonal elements of the resolvent converge almost surely. It follows that $m_{\wh{Q}_n}(z)$ also converges almost surely, with limit
\begin{align*}
m_{\wh{Q}_n}(z) 
& \to \int \frac{\tau}{\tau(1 - \gamma) - z - \tau z \gamma m(z)} \,dH(\tau) \\
& = -\frac{1}{\gamma z}\Big(z + \frac{1}{v(z)}\Big) \tag{by~\eqref{eqn:silverstein}}\\
& = -\frac{1}{\gamma}\Big(1 + \frac{1}{zv(z)}\Big).
\end{align*}

\iffalse
The key is to analyze $\bS \bR$. First, we show that $\frac{1}{p}\tr(\bS \bR)$ can be related to the quantity of interest. Notice that $\frac{1}{p}\tr(\bS \bR)$ can be written as the following sum of quadratic forms:
\begin{equation}
	\label{eqn:spectral-measure-Sigma-1}
	\frac{1}{p}\tr(\bS \bR) = \frac{1}{pn} \sum_{k = 1}^{n} \bx_k^{\top} \bR \bx_k.
\end{equation}
We analyze each quadratic form using leave-one-out arguments. Introduce the leave-one-out matrices
$$
\bS_k = \frac{1}{n} \sum_{i \neq k} \bx_i \bx_i^{\top}, \quad \bR_k = (\bS_k - z \bI)^{-1}.
$$
By the resolvent identity, 
$$
\bR_k - \bR = \bR_k(\frac{1}{n} \bx_k \bx_k^{\top}) \bR
$$
and thus
$$
\bx_k^{\top} \bR \bx_k = \frac{\bx_k^{\top} \bR_k \bx_k}{1 + (1/n)\bx_k^{\top} \bR_k \bx_k}.
$$
Arguing as in the preceding section, we have that $\bx_k^{\top} \bR_k \bx_k \approx \tr(\bSigma \bR_k) \approx \tr(\bSigma \bR)$. [The approximations are up to asymptotically negligible terms as $n,p \to \infty$.] Plugged back into~\eqref{eqn:spectral-measure-Sigma-1},
$$
\frac{1}{p}\tr(\bS \bR) \approx \frac{\tr(\bSigma \bR)}{p(1 + (1/n) \tr(\bSigma \bR))} = \frac{(1/p) \tr(\bSigma \bR)}{1 + (\gamma/p) \tr(\bSigma \bR)}.
$$
On the other hand, by the identity $\bS = \bS - z\bI + z \bI$, we have
$$
\bS \bR = \bI  + z \bR \Longrightarrow \sum_{i = 1}^{n} \bx_i \bx_i^{\top} \bR = \bI  + z \bR.
$$ 
Therefore, $(1/p) \tr(\bS \bR) = 1 + z/p\tr(\bR) = 1 + z m_{n}(z)$. We conclude that 
$$
\frac{m_{\hat{Q}_n}(z)}{1 + \gamma m_{\hat{Q}_n}(z)} \approx 1 + z m_{n}(z),
$$ 
where $m_n(z)$ is the Stieltjes transform of $\bS$. In the limit:
$$
\frac{m_{Q}(z)}{1 + \gamma m_{Q}(z)} \approx 1 + z m_{\gamma,H}(z),
$$
which can be rearranged to 
$$
m_{Q}(z) = -\frac{1}{\gamma}\bigg(1 + \frac{1}{z v_{\gamma,H}(z)}\bigg).
$$
\fi

\subsection{Spectral measure of $\bSigma$}
Having established convergence of the Stieltjes transform, we now invert the Stieltjes transform to recover the limiting measure $Q = Q_{\gamma,H}$. A calculation shows that that the support of $Q$ is equal to the support of $F$. At $\theta \in \supp(F), \theta > 0$, we have that $Q$ is continuous with density
\begin{equation}
	\label{eqn:spectral-measure-covariance-density}
	q(\theta) = \frac{f_{\gamma,H}(\theta)}{\theta |v_{\gamma,H}(z)|^2}.
\end{equation}
The measure also has an atom at $\theta = 0$ of weight
\begin{equation}
	\label{eqn:spectral-measure-covariance-atom-at-zero}
	Q_{\gamma,H}(0) = - \lim_{\eta \to 0} \eta i m_Q(\eta i) = 
	\begin{dcases*}
		0, & \quad \textrm{$\gamma < 1 - H(0)$}, \\
		\frac{1}{\gamma v_0}, & \quad \textrm{$\gamma > 1 - H(0)$}.
	\end{dcases*} 
\end{equation}
where $v_0$ is defined in~\eqref{eqn:stieltjes-at-zero}.

\section{Spectral measure of eigenvectors}
Define the following population and sample spectral distributions of a vector $\bbeta \in \Reals^p$:
$$
\wh{G}_n(\tau) := \frac{1}{\|\bbeta\|^2} \sum_{j = 1}^{p} (\bbeta_j^{\top} \bw_j)^2 \1\{\lambda_i(\bSigma) \leq \tau\}, \quad \wh{B}_n(\tau) := \frac{1}{\|\bbeta\|^2} \sum_{j = 1}^{p} (\bbeta_j^{\top} \bv_j)^2 \1\{\lambda_i(\bS) \leq \tau\}
$$ 
Obviously, taking $\bbeta = \bw$ means that $\wh{B}_n$ describes the correlation between the leading eigenspaces $\mathrm{span}\{\bv_1,\ldots,\bv_j\}$ and a true eigenvector of $\bSigma$. (Although, notice that the limit of $\wh{B}_n$ does not contain enough microscopic information to describe the asymptotic correlation between a population eigenvector and an individual sample eigenvector.)

The limit of $\wh{B}_n$, as $n,p \to \infty, p/n \to \gamma \in (0,\infty)$ and $\hat{G}_n \convweak G$, can be computed by finding the limit of the Stieltjes transform of $\hat{B}_n$, 
\begin{equation}
	\label{eqn:spectral-density-stieltjes}
	m_{\hat{B}_n}(\tau) = \frac{1}{\|\bbeta\|^2} \dotp{\bbeta}{\bR(z) \bbeta},
\end{equation}
and applying the Stieltjes inversion formula. % The fundamental ideas used to derive the limit of the Stieltjes transform of $m_{\wh{B}_n}$ are largely the same as for eigenvalues, but the calculations are a bit more tedious. 

\subsection{Convergence of the Stieltjes transform~\eqref{eqn:spectral-density-stieltjes}}
\label{subsec:convergence-of-eqn:spectral-density-stieltjes}
As in Section~\ref{subsec:proof-of-eqn:silverstein}, we do not provide a rigorous proof. Instead we only show convergence of expectation $\E m_{\hat{B}_n}(\tau) \to m_{B}(\tau)$, under the assumption that $m_{\hat{B}_n}(\tau)$ and the diagonal resolvent terms $\bR(z)_{jj}$ concentrate around their expectations.

We divide the expectation of interest into two terms, the on-diagonal and off-diagonal terms:
\begin{equation*}
	m_{\hat{B}_n}(z) = \frac{1}{p} \sum_{j = 1}^{p} \bbeta_j^2 [\bR(z)]_{jj} + \frac{1}{p} \sum_{i \neq j}^{p} \bbeta_i \bbeta_j [\bR(z)]_{ij}.
\end{equation*}
We have already computed the almost sure limit of the on-diagonal term in~\eqref{eqn:resolvent-diagonal}; by assumption, this also represents the limit of the expectation of the on-diagonal term. For the off-diagonal terms, by block-matrix inversion,
\begin{align*}
	[\bR(z)]_{1\cdot} 
	& = -[\bR(z)]_{11} \bS_{1\cdot} \bR_{-1}(z)  = -[\bR(z)]_{11} \frac{\bX_{\cdot 1}^{\top} \bX_{\cdot -1}}{n} \bR_{-1}(z). 
\end{align*}
As $\bSigma$ is diagonal this last term has expectation $0$, conditional on $\bX_{\cdot -1}$ and therefore marginally: that is
\begin{equation}
	\label{eqn:off-diagonal-resolvent}
	\mathbb{E}[\bR(z)_{ij}|\bX_{\cdot-i}] = 0, \quad \textrm{for all $i \neq j$.}
\end{equation}
Thus the expectation of the off-diagonal term is $0$, and
\begin{equation*}
	\E m_{\hat{B}_n}(z) \sim \frac{1}{p}\sum_{j = 1}^{p} \bbeta_j^2 [\bR(z)]_{jj} \to \int \frac{1}{\tau - z - \gamma \tau(1 + z m(z))} \,dG(\tau) = -\frac{1}{z}\int\frac{1}{\tau v(z) + 1} \,dG(\tau).
\end{equation*}
It is indeed the case that $m_{\hat{B}_n}(z)$ converges almost surely as $n,p \to \infty, n/p \to \gamma \in (0,\infty)$, with limit
\begin{equation}
	\label{eqn:stieljtes-limiting-spectral-density}
	m_{B}(z) = -\frac{1}{z}\int\frac{1}{\tau v(z) + 1} \,dG(\tau).
\end{equation}

\subsection{Spectral density of $\bbeta$}
Finally, applying the inversion formula to the limiting Stieltjes transform as given in~\eqref{eqn:stieljtes-limiting-spectral-density}, the density of $f_{B}(\theta)$ at any $\theta \in \supp(B)$ is given by
$$
f_{B}(\theta) = \frac{1}{\pi}\lim_{\varepsilon \to} \Im(m_B(\theta + \varepsilon i)) = \frac{1}{\theta} \int \frac{\tau \gamma f_{\gamma,H}(\theta)}{|1 + \tau v(\theta)|^2} \,dG(\tau),
$$
assuming this limit is finite. If the limit is infinite, then $\theta$ is a potential atom of $B$. 

Suppose $\bbeta$ is the $\alpha p$th eigenvector of $\bSigma$. Then as $p \to \infty$, $\hat{G}_n(\tau) \to \delta_{Q_{\alpha}}$, where $Q_{\alpha,\gamma,H}$ is the $(1 - \alpha)$th quantile of $F_{\gamma,H}$. We compute the empirical spectral density of a population eigenvector as
$$
f_{B}(\theta) = \frac{1}{\theta} \frac{Q_{\alpha,\gamma,H} \gamma f_{\gamma,H}(\theta)}{|1 + Q_{\alpha,\gamma,H} v(\theta)|^2},
$$
assuming again that this is finite.

\section{Bias of Principal Component Regression}
Introduce the two-dimensional measure,
\begin{equation}
	\label{eqn:}
	K_n(\theta,\varphi) := \frac{1}{\|\bbeta\|^2} \sum_{i,j = 1}^{p} \dotp{\bbeta}{\bv_i} \dotp{\bbeta}{\bv_j} \dotp{\bv_i}{\bSigma \bv_{j}} \1(\lambda_i(\bS) \leq \theta, \lambda_j(\bS) \leq \varphi).
\end{equation}
Evaluated on its diagonal $K_n(\theta,\theta)$, the measure corresponds to the out-of-sample prediction bias of Principal Component Regression. We compute limit of $K_n(\theta,\varphi)$ by calculating the limit of its Stieltjes transform -- the multi-resolvent trace
\begin{equation}
	\label{eqn:2d-stieltjes-transform}
	m_{K_n}(z,w) := \frac{1}{\|\bbeta\|^2} \dotp{\bbeta}{\bR(z) \bSigma \bR(z') \bbeta},
\end{equation}
and applying a suitable inversion formula.

\subsection{Convergence of the two-dimensional Stieltjes transform~\eqref{eqn:2d-stieltjes-transform}}
As in Sections~\ref{subsec:proof-of-eqn:silverstein} and~\ref{subsec:convergence-of-eqn:spectral-density-stieltjes}, we will only compute the expectation, and not formally prove concentration of the functional~\eqref{eqn:2d-stieltjes-transform} about its expectation.

Begin by writing $m_{K_n}(z,w)$ as a iterated sum, and strategically grouping terms:
\begin{align*}
	m_{K_n}(z,w) 
	& = \frac{1}{\|\bbeta\|^2}\sum_{i,j = 1}^{p} \bbeta_i \bbeta_j (\bR(z) \bSigma \bR(w))_{ij} \\
	& = \frac{1}{\|\bbeta\|^2}\sum_{i,j,k,l = 1}^{p} \bbeta_i \bbeta_j \bR(z)_{il} \bR(z)_{kl} \bSigma_{kl} \\
	& = \frac{1}{\|\bbeta\|^2}\sum_{i,j = 1}^{p} \bbeta_i \bbeta_j \bigg(\sum_{k = 1}^{p} \tau_k \bR(z)_{ik} \bR(z)_{kj}\bigg) \\
	& = \frac{1}{\|\bbeta\|^2}\sum_{j = 1}^{p} \tau_j \bbeta_j^2 \Big(\bR(z)_{jj} \bR(w)_{jj} \Big) + \frac{1}{\|\bbeta\|^2} \sum_{j = 1}^{p} \bbeta_i^2 \Big(\sum_{k \neq i}^{p} \tau_k \bR(z)_{ik} \bR(w)_{ki}\Big) + \frac{1}{\|\bbeta\|^2} \sum_{i \neq j}^{p} \bbeta_i \bbeta_j \Big(\sum_{k = 1}^{p} \bR(z)_{ik} \bR(z)_{kj}\Big) \\
	& =: A_1 + A_2 + A_3.
\end{align*}
We will separately compute the expectation of each term. 

For Term 1, we have from~\eqref{eqn:almost-sure-convergence-diagonal-resolvent} that the diagonal terms of the resolvent concentrate almost surely:
$$
\bR(z)_{ii} \to \bigg(\tau_i - z + \tau_i \gamma(1 + z \gamma m(z))\bigg)^{-1} = \frac{-1}{z(1 + \tau_i v(z))}.
$$
Therefore
$$
\mathbb{E}[A_1] \to \frac{1}{zw} \int \frac{\tau}{(1 + \tau v(z))(1 + \tau v(w))} \,dG(\tau).
$$
The expectation of Term 3 is zero: by~\eqref{eqn:off-diagonal-resolvent},
\begin{equation*}
	\mathbb{E}[A_3] = \frac{1}{\|\bbeta\|^2} \sum_{i \neq j}^{p} \bbeta_i \bbeta_j \Big(\sum_{k \neq i}^{p} \mathbb{E}\Big[\mathbb{E}[\bR(z)_{ik}|\bX_{-i}] \bR(z)_{kj}\Big] + \mathbb{E}\Big[\mathbb{E}[\bR(z)_{jk}|\bX_{-j}] \bR(z)_{ii}\Big]\Big) = 0.
\end{equation*}
This leaves us with Term 2. We can express the inner sum as 
\begin{equation}
	\label{eqn:pf-2d-Stieltjes-transform-1}
	\sum_{k \neq i}^{p} \tau_k \bR(z)_{ik} \bR(w)_{ki} = \frac{1}{n^2}[\bR(z)]_{ii} [\bR(w)]_{ii} \bX_{\cdot i}^{\top} \bX_{\cdot -i} \bR_{-i}(z) \bSigma \bR_{-i}(w) \bX_{-i \cdot} \bX_{\cdot i}^{\top}.
\end{equation}
Focusing on the second part of this expression, we have that conditional on $\bX_{-i}$,
\begin{align}
	\label{eqn:pf-2d-Stieltjes-transform-2}
	\frac{1}{n^2}\mathbb{E}[\bX_{\cdot i}^{\top} \bX_{\cdot -i} \bR_{-i}(z) \bSigma \bR_{-i}(w) \bX_{-i \cdot} \bX_{\cdot i}^{\top}|\bX_{-i}] 
	& = \frac{\tau_i}{n^2} \tr\bigg(\bX_{\cdot -i} \bR_{-i}(z) \bSigma \bR_{-i}(w) \bX_{-i \cdot}\bigg) \nonumber \\
	& = \frac{\tau_i}{n} \tr\bigg(\bS_{-i} \bR_{-i}(z) \bSigma \bR_{-i}(w)\bigg) \nonumber \\
	& = \frac{\tau_i}{n}\bigg\{\tr\big(\bSigma \bR_{-i}(w)\big) + z \tr\big(\bR_{-i}(z) \bSigma \bR_{-i}(w) \big)\bigg\}.
\end{align}
We have already computed the almost sure limit of $\frac{1}{n}\tr\big(\bSigma \bR_{-i}(w)\big)$ in Section~\ref{sec:spectral-measure-population-covariance}, where we showed that 
\begin{equation*}
	\frac{1}{n}\tr\big(\bSigma \bR_{-i}(w)\big) \to -\Big(1 + \frac{1}{wv(w)}\Big).
\end{equation*}
On the other hand, for the second part of~\eqref{eqn:pf-2d-Stieltjes-transform-2}, we have that for all $z \neq w$,
\begin{align*}
	\frac{1}{n}\tr\big(\bR_{-i}(z) \bSigma \bR_{-i}(w) \big) 
	& = 
	\frac{1}{n}\tr\big(\bR_{-i}(w) \bR_{-i}(z) \bSigma \big) \\
	& = 
	\frac{1}{n(z - w)}\tr\big((\bR_{-i}(z)  - \bR_{-i}(w)) \bSigma \big) \\
	& \to -\frac{1}{(z - w)}\Big(\frac{1}{zv(z)} - \frac{1}{wv(w)}\Big).
\end{align*}
Summing these two terms together, some routine algebra shows that the almost sure limit of~\eqref{eqn:pf-2d-Stieltjes-transform-2} is
\begin{equation*}
	-\tau_i \Big\{1 + \frac{K(z,w)}{v(z)v(w)}\Big\}, \quad \textrm{where}\quad K(z,w) := \frac{v(z) - v(w)}{z - w}, \textrm{for all $z \neq w$.}
\end{equation*}
Plugging this back into~\eqref{eqn:pf-2d-Stieltjes-transform-1} gives that the almost sure limit of~\eqref{eqn:pf-2d-Stieltjes-transform-1} is
\begin{equation*}
	-\frac{\tau_i}{zw(1 + \tau_i v(z)) (1 + \tau_i v(w))} \Big\{1 + \frac{K(z,w)}{v(z)v(w)}\Big\}.
\end{equation*}
Finally, summing over all $i = 1,\ldots,p$ gives the asymptotic limit of $\mathbb{E}[A_2]$:
\begin{equation*}
	\mathbb{E}[A_2] \to -\frac{1}{zw}\Big\{1 + \frac{K(z,w)}{v(z)v(w)}\Big\}\int \frac{\tau}{(1 + \tau v(z)) (1 + \tau v(w))} \,dG(\tau).
\end{equation*}
Adding this to $\mathbb{E}[A_1]$ and $\mathbb{E}[A_3]$ (the last being $0$), we conclude that 
\begin{equation*}
	\E[m_{K_n}(z,w)] \to -\frac{K(z,w)}{v(z) v(w)} \int \frac{\tau}{(1 + \tau v(z)) (1 + \tau v(w))} \,dG(\tau).
\end{equation*}
This is indeed the Stieltjes transform $m_{K}(z,w)$ of the limiting measure $K$ of $K_n$.

\section{Examples}

\subsection{Identity covariance}

The most basic case is $\bSigma = \bI, \gamma < 1$, $\hat{H}_n(\tau) = H_1(\tau) = \1\{\tau \leq 1\}$. In this case $\hat{F}_n(t)$ converges weakly to the {\bf Marchenko-Pastur distribution} $F_{\gamma}(t)$. The Marchenko-Pastur distribution has a density $f_{\gamma}(t)$ supported on $[t^{-},t^{+}] := [1 - \sqrt{\gamma}, 1 + \sqrt{\gamma}]$, with 
$$
f_{\gamma}(t) := \frac{\sqrt{(t^{+} - t)(t - t^{-})}}{2 \pi \gamma t}, \quad \textrm{for $t \in [t^{-},t^{+}]$.}
$$
If additionally $\gamma > 1$, then $F_{\gamma}$ has an atom at $0$ of magnitude $1 - 1/\gamma$. 

To show this, notice that when $H = \delta_1$ the self-consistency equation~\eqref{eqn:silverstein} reduces to
$$
z = -\frac{1}{v(z)} + \frac{\gamma}{1 + v(z)}.
$$
Using the definition of $m = \gamma v - (1 - \gamma)/z$, this can be rearranged to 
$$
m(z)= \frac{1}{(1 - \gamma) - \gamma m z - z}.
$$
Taking the resulting quadratic equation and solving for $m$ yields two solutions, 
$$
\frac{(1 - \gamma - z) \pm \sqrt{(1 + \gamma - z)^2 - 4 \gamma}}{2 \gamma z}.
$$
If we adopt the usual convention that $\sqrt{z} \in \mathbb{C}_{+}$ refers to the positive branch of a complex number, then recalling that $m(z) \in \mathbb{C}_{+}$, it follows that 
\begin{align*}
m_{\gamma}(z) 
& = \frac{(1 - \gamma - z) + \sqrt{(1 + \gamma - z)^2 - 4 \gamma}}{2 \gamma z} \\
& = \frac{(1 - \gamma - z) + \sqrt{( (1 + \sqrt{\gamma})^2 - z)((1 - \sqrt{\gamma})^2 - z)}}{2 \gamma z}.
\end{align*}
From here we can compute the density of the Marchenko-Pastur distribution by applying the Stieltjes inversion formula:
\begin{align*}
f_{\gamma}(t) 
& = \frac{1}{\pi}\lim_{\varepsilon \to 0} \Im m_{\gamma}(t + i \varepsilon) \\
& = \frac{1}{\pi} \Im \frac{\sqrt{( (1 + \sqrt{\gamma})^2 - t)((1 - \sqrt{\gamma})^2 - t)}}{2 \gamma t} \\
& = \frac{1}{\pi} \frac{\sqrt{( (1 + \sqrt{\gamma})^2 - t)(t - (1 - \sqrt{\gamma})^2)}}{2 \gamma t}.
\end{align*}
Meanwhile,
$$
m_{B_{\gamma}}(\theta) = -\frac{1}{\theta(v_{\gamma}(\theta) + 1)} \textcolor{red}{= m_{\gamma}(\theta)}, 
$$
showing that $B_{\gamma} = F_{\gamma}$, as one would expect.

\subsection{Spiked covariance, one-spike}
The next case that will be considered is the one-spiked covariance model where $\bSigma = (\ell - 1) \bbeta \bbeta^{\top} + \bI$ for some $\ell \geq 1$, where $\|\bbeta\|^2 = 1$ and $\bbeta, \ell$ are fixed in $n$. As in the identity covariance model $\hat{H}_n(\tau) \to H_1(\tau)$, and so $\hat{F}_n(t) \to F_{\gamma}(t)$. 

Unlike in the identity covariance case, $G_n \to \delta_{\ell}$. Therefore the Stieltjes transform of the empirical spectral distribution of $\beta$ has limit
\begin{equation}
\label{eqn:spiked-covarinace-limiting-stieltjes}
m_{B_{\ell}}(\theta) = -\frac{1}{\theta} \cdot \frac{1}{\ell v(\theta) + 1}. 
\end{equation}
Clearly $m_B(\theta) \in \Reals$ if and only if $v(\theta) \in \Reals$ and $v(\theta) \neq -(1/\ell)$. Since singletons do not contribute to the $\interior(\supp(B))$, it follows that 
$$
\interior(\supp(B)) = \interior(\supp(F_\gamma)) = (1 - \sqrt{\gamma}, 1 + \sqrt{\gamma}).
$$
For $\theta \in (1 - \sqrt{\gamma}, 1 + \sqrt{\gamma})$, the density of $B_{\ell}$ is given by the Stieljes inversion formula. We work through the tedious algebraic details for completeness. First of all, the companion transform can be written as 
\begin{align*}
v_{\gamma}(\theta) 
& = \gamma m_{\gamma}(\theta) - \frac{1 - \gamma}{\theta} \\
& = \gamma \bigg\{\frac{(1 - \gamma - \theta) + \sqrt{(1 + \gamma - \theta)^2 - 4 \gamma}}{2 \gamma \theta}\bigg\}- \frac{(1 - \gamma)}{\theta} \\
& = -\frac{1}{2} - \frac{(1 - \gamma)}{2 \theta} + \frac{\sqrt{(1 + \gamma - \theta)^2 - 4 \gamma}}{2\theta}.
\end{align*}
A calculation then gives
\begin{align*}
|1 + \ell v(\theta)|^2 
& = 
\bigg|1 - \frac{\ell}{2} - \frac{(1 - \gamma)\ell}{2 \theta} + \frac{\ell\sqrt{(1 + \gamma - \theta)^2 - 4 \gamma}}{2\theta}\bigg|^2 \\
& = 
\bigg(1 - \frac{(1 - \gamma + \theta)\ell}{2 \theta}\bigg)^2 - \frac{\ell^2\Big((1 + \gamma - \theta)^2 - 4 \gamma\Big)}{4\theta^2} \\
& = 
1 - \frac{(1 - \gamma + \theta)\ell}{ \theta} + \frac{\ell^2(1 - \gamma + \theta)^2- (1 + \gamma - \theta)^2}{4\theta^2} + \frac{\ell^2\gamma}{\theta^2} \\
& = 1 - \frac{(1 - \gamma + \theta)\ell}{ \theta} + \frac{\ell^2}{\theta}. 
\end{align*}
Therefore, 
\begin{align*}
f_{B_{\ell}}(\theta) & = \frac{1}{\theta} \frac{\ell \gamma f_{\gamma}(\theta)}{|1 + \ell v(\theta)|^2}  = \frac{\gamma \ell}{\theta + \ell(\gamma - 1 - \theta) + \ell^2} f_{\gamma}(\theta), \quad \textrm{for $\theta \in (1 - \sqrt{\gamma}, 1 + \sqrt{\gamma}).$}
\end{align*}
It can be shown that this holds at the boundaries $\theta = \pm(1 - \sqrt{\gamma})$ as well.

Finally, it can be seen from~\eqref{eqn:spiked-covarinace-limiting-stieltjes} that there is a potential atom in $B_{\ell}$ if there exists $\theta \in \Reals_{+}$ such that $v_{\gamma}(\theta) = -1/\ell$, in other words, if $-1/\ell \in \range(v_{\gamma})$. As detailed in Section~\ref{sec:inverse-companion-transform}, the range of $v_{\gamma}$ can be identified from the inverse of the Stieltjes transform. The inverse of the Stieltjes transform is given in~\eqref{eqn:inverse-fixed-point}, and in the special case where $H = \delta_1$:
$$
v_{\gamma}^{-1}(z) = -\frac{1}{z} + \frac{\gamma}{1 + z}.
$$
It follows that
$$
(v_{\gamma}^{-1})'(z) = \frac{1}{z^2} - \frac{\gamma}{(1 + z)^2},
$$
which is strictly positive when $-1/z > (1 + \sqrt{\gamma})$. In other words, $-1/\ell \in \range(v_{\gamma})$ if $\ell > (1 + \sqrt{\gamma})$. Thus, if $\ell > (1 + \sqrt{\gamma})$, there exists an atom at 
$$
v_{\gamma}^{-1}(-1/\ell) = \ell + \frac{\gamma \ell}{\ell - 1} =: \theta_l,
$$
and the weight of the atom is given by
\begin{align*}
\lim_{\varepsilon \to 0} - \varepsilon i m_{B}(\theta + \varepsilon i) 
& = \frac{-1}{\ell \theta_\ell v_{\gamma}'(\theta_l)} = \frac{(v_{\gamma}^{-1})'(-1/\ell)}{\ell \theta_{\ell}} \\
& = \frac{1 - \gamma/(\ell - 1)^2}{1 + \gamma/(\ell - 1)}.
\end{align*}


	
	
\end{document}