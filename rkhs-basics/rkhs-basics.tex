\documentclass{article}

% Packages
\usepackage[utf8]{inputenc} % allow utf-8 input
\usepackage[T1]{fontenc}    % use 8-bit T1 fonts
\usepackage{booktabs}       % professional-quality tables
\usepackage{nicefrac}       % compact symbols for 1/2, etc.
\usepackage{microtype}      % microtypography
% \usepackage{times}             % times font
\usepackage{mathrsfs}      % Added by Alden, for script font.
\usepackage{ragged2e}     % Added by Alden, for no indent.
\usepackage{parskip}        % Added by Alden, for skips between paragraphs.

\usepackage[round]{natbib}
\usepackage{amssymb,amsmath,amsthm,bbm}
\usepackage[margin=1in]{geometry}
\usepackage{verbatim,float,url,dsfont}
\usepackage{graphicx,subcaption,psfrag} % Alden changed subfigure to subcaption
\usepackage{algorithm,algorithmic}
\usepackage{mathtools}
\usepackage[shortlabels]{enumitem}       % Alden added shortlabels option
\usepackage[colorlinks=true,citecolor=blue,urlcolor=blue,linkcolor=blue]{hyperref}
\usepackage{multirow}

% Theorems and such
\newtheorem{theorem}{Theorem}
\newtheorem{lemma}{Lemma}
\newtheorem{corollary}{Corollary}
\newtheorem{proposition}{Proposition}
\theoremstyle{definition}
\newtheorem{remark}{Remark}
\newtheorem{definition}{Definition}
\newtheorem{example}{Example} % Added by Alden
\newtheorem{conjecture}{Conjecture} % Added by Alden

% Assumption
\newtheorem*{assumption*}{\assumptionnumber}
\providecommand{\assumptionnumber}{}
\makeatletter
\newenvironment{assumption}[2]{
  \renewcommand{\assumptionnumber}{Assumption #1#2}
  \begin{assumption*}
  \protected@edef\@currentlabel{#1#2}}
{\end{assumption*}}
\makeatother

% Widebar
\makeatletter
\newcommand*\rel@kern[1]{\kern#1\dimexpr\macc@kerna}
\newcommand*\widebar[1]{%
  \begingroup
  \def\mathaccent##1##2{%
    \rel@kern{0.8}%
    \overline{\rel@kern{-0.8}\macc@nucleus\rel@kern{0.2}}%
    \rel@kern{-0.2}%
  }%
  \macc@depth\@ne
  \let\math@bgroup\@empty \let\math@egroup\macc@set@skewchar
  \mathsurround\z@ \frozen@everymath{\mathgroup\macc@group\relax}%
  \macc@set@skewchar\relax
  \let\mathaccentV\macc@nested@a
  \macc@nested@a\relax111{#1}%
  \endgroup
}
\makeatother

% Min and max
\newcommand{\argmin}{\mathop{\mathrm{argmin}}}
\newcommand{\argmax}{\mathop{\mathrm{argmax}}}
\newcommand{\minimize}{\mathop{\mathrm{minimize}}}
\newcommand{\st}{\mathop{\mathrm{subject\,\,to}}}
\DeclareMathOperator*{\esssup}{ess\,sup} % Added by Alden

% Shortcuts
\def\R{\mathbb{R}}
\def\C{\mathbb{C}}
\def\E{\mathbb{E}}
\def\P{\mathbb{P}}
\def\T{\mathsf{T}}
\def\Cov{\mathrm{Cov}}
\def\Var{\mathrm{Var}}
\def\half{\frac{1}{2}}
\def\tr{\mathrm{tr}}
\def\df{\mathrm{df}}
\def\dim{\mathrm{dim}}
\def\col{\mathrm{col}}
\def\row{\mathrm{row}}
\def\nul{\mathrm{null}}
\def\rank{\mathrm{rank}}
\def\nuli{\mathrm{nullity}}
\def\spa{\mathrm{span}}
\def\sign{\mathrm{sign}}
\def\supp{\mathrm{supp}}
\def\diag{\mathrm{diag}}
\def\aff{\mathrm{aff}}
\def\conv{\mathrm{conv}}
\def\dom{\mathrm{dom}}
\def\hy{\hat{y}}
\def\hf{\hat{f}}
\def\hmu{\hat{\mu}}
\def\halpha{\hat{\alpha}}
\def\hbeta{\hat{\beta}}
\def\htheta{\hat{\theta}}
\def\cA{\mathcal{A}}
\def\cB{\mathcal{B}}
\def\cD{\mathcal{D}}
\def\cE{\mathcal{E}}
\def\cF{\mathcal{F}}
\def\cG{\mathcal{G}}
\def\cK{\mathcal{K}}
\def\cH{\mathcal{H}}
\def\cI{\mathcal{I}}
\def\cL{\mathcal{L}}
\def\cM{\mathcal{M}}
\def\cN{\mathcal{N}}
\def\cP{\mathcal{P}}
\def\cS{\mathcal{S}}
\def\cT{\mathcal{T}}
\def\cW{\mathcal{W}}
\def\cX{\mathcal{X}}
\def\cY{\mathcal{Y}}
\def\cZ{\mathcal{Z}}

%%% Begin Alden's additions
\newcommand{\Ebb}{\mathbb{E}}
\newcommand{\Pbb}{\mathbb{P}}
\newcommand{\dotp}[2]{\langle #1, #2 \rangle}
\newcommand{\wt}[1]{\widetilde{#1}}
\newcommand{\wh}[1]{\widehat{#1}}
\newcommand{\mc}[1]{\mathcal{#1}}
\newcommand{\Reals}{\mathbb{R}} % Same thing as Ryan's \R
\newcommand{\Rd}{\Reals^d}
\newcommand{\wb}[1]{\widebar{#1}}
\newcommand{\floor}[1]{\left\lfloor #1 \right\rfloor}
\newcommand{\ceil}[1]{\left\lceil #1 \right\rceil}
\newcommand{\1}{\mathbf{1}}
\newcommand{\bj}{{\bf j}}
\newcommand{\restr}[2]{\ensuremath{\left.#1\right|_{#2}}}
\newcommand{\TV}{\mathrm{TV}}

\DeclareFontFamily{U}{mathx}{\hyphenchar\font45}
\DeclareFontShape{U}{mathx}{m}{n}{<-> mathx10}{}
\DeclareSymbolFont{mathx}{U}{mathx}{m}{n}
\DeclareMathAccent{\wc}{0}{mathx}{"71}


%%% End Alden's Additions
\usepackage{lmodern}

\def\BL{\mathrm{BL}}


\begin{document}
	\title{{ \bf Representer theorem for thin-plate splines}}
	\author{Alden Green}
	\date{}
	\maketitle
	\RaggedRight
	
Given pairs $(x_i,y_i) \in \Reals^{d + 1}$, the thin-plate spline problem is to minimize
\begin{equation}
	\label{eqn:tps}
	\min \frac{1}{n}\sum_{i = 1}^{n}(y_i - f(x_i))^2 + \lambda J_d^{m}(f),
\end{equation}
where $J_d^{m}(f) = \sum_{|\alpha| = m} \frac{m!}{\alpha!} \int_{\Rd} (D^{\alpha}f(x))^2 \,dx$ is a derivative-based penalty of roughness.

Thin-plate splines were originally proposed by~\citet{duchon1977}, though in the context of interpolation rather than penalized regression. Duchon showed that when $m > d/2$ there is a representer theorem for the thin-plate spline problem~\eqref{eqn:tps} that allows it to be recast as a generalized ridge regression. The result is analogous to the representer theorem for Reproducing Kernel Hilbert Spaces (RKHS), but with the added subtlety that the regularizer $J_d^m(f)$ has a non-trivial null space. For this reason the representer theorem for~\eqref{eqn:tps} does not immediately follow from results for RKHS. Instead~\eqref{eqn:tps} must be derived separately, and the proof must correctly handle the null space of $J_d^m(f)$.

The proof of Duchon relies heavily on Fourier transforms and is overall a little technical. Several subsequent papers~\citet{meinguet1979,wendelburger1981} and the book~\citet{gu2013smoothing} give a refined and simplified analysis, but the first reference still requires an appreciable level of mathematical sophistication and the latter two are a little fast. In this note I explicitly state the TPS representer theorem (Theorem~\ref{thm:tps-representer} below) and walk through a simple version of the proof. 

\section{The theorem}

The solution to~\eqref{eqn:fundamental-solution} is made up of radial basis functions (RBFs) and polynomials. The RBFs we care about are translates of the fundamental solution (also known as a Green's function) $E$ of the polyharmonic equation
\begin{equation}
	\label{eqn:fundamental-solution}
	(-1)^m\Delta^{m} E = \delta_0.
\end{equation}
In equation~\eqref{eqn:fundamental-solution} the operator $\Delta^{m}$ is the $k$th-iterated Laplacian -- defined recursively -- and $\delta_0$ is the Dirac impulse. The fundamental solution has the explicit form
\begin{equation*}
	E(r) =  c_d 
	\begin{dcases}
		r^{2m - d} \ln r, \quad & \textrm{if $2m > d$ and $d$ is even} \\
		r^{2m - d}, \quad & \textrm{otherwise.}
	\end{dcases}
\end{equation*}
Here $c_d$ is some complicated constant I won't bother writing out. The RBF at point $x_i$ is given by $E(x_i,\cdot) := E(|x_i - \cdot|)$. 

As mentioned, the null space $\mc{N}$ of $J_d^m(f)$ is non-trivial and in fact consists of all polynomials of degree at most $m - 1$. This is a $M = {m + d - 1\choose d}$ dimensional vector space. Let $\phi_{1},\ldots,\phi_M$ be an arbitrary basis of $\mc{N}$. Let ${\bf \Phi} \in \Reals^{n \times m}$ have entries ${\bf \Phi}_{ij} = \phi_j(x_i)$ and ${\bf E} \in \Reals^{n \times n}$ have entries ${\bf E}_{ij} = E(x_i,x_j)$. 
\begin{theorem}
	\label{thm:tps-representer}
	Suppose $m > d/2$. Then the solution to~\eqref{eqn:tps} is well-defined and can be written in the form 
	$$f_{\lambda}(x) = \sum_{\nu = 1}^{M} c_{\nu} \phi_{\nu}(x) + \sum_{i = 1}^{n} d_i E(x_i,x),$$ 
	where additionally $\sum_{i = 1}^{n} d_i \phi_{\nu}(x_i) = 0$ for each $j$. The optimization can be rewritten as the \emph{generalized ridge} problem
	\begin{equation}
		\label{eqn:tps-ridge}
		\min_{c,d} \|y - {\bf \Phi} c - {\bf E} d\|^2 + \lambda d^{\top} {\bf E} d,
	\end{equation}
	which has a unique solution so long as $\mathrm{rank}(x_1,\ldots,x_n) \geq M$.
\end{theorem}

\section{Formalizing thin-plate splines}
Formally speaking, to make sense of problem~\eqref{eqn:tps} one needs a domain. The first thought is to use the Sobolev space $W^{m,2}(\Rd)$ which is the space of $m$-times weakly differentiable functions for which $D^{\alpha}f \in L^2(\Rd)$ for all $|\alpha| \leq m$. Technically speaking the Sobolev space is made up of equivalence classes of functions that agree up to sets of measure zero, but when $m > d/2$ each equivalence class contains a continuous representative (by the Sobolev Embedding Theorem) and so we can restrict our attention to continuous functions.  

A more fundamental concern is that polynomials are not in $W^{m,2}(\Rd)$ and so clearly that cannot be the right space to search for the solution. This issue is cleared up by using the Beppo Levi space, which consists of functions
\begin{equation}
	\mathrm{BL}_{m}(\Rd) = \{f \in C(\Rd),~~ D^{\alpha} f~\textrm{exists and}~ D^{\alpha}f  \in L^2(\Rd)~~\textrm{for all $\alpha = |m|$} \},
\end{equation}
and is equipped with the semi-inner product 
\begin{equation}
	(u,v)_{\mathrm{BL}_{m}(\Rd)} := \sum_{|\alpha| = m} \frac{m!}{\alpha!} (D^{\alpha}u, D^{\alpha}v)_{L^2}.
\end{equation}
For simplicity write $(u,v)_{m} = (u,v)_{\mathrm{BL}_{m}(\Rd)}$. Throughout, the minimum in~\eqref{eqn:tps} should be interpreted as being over all functions $f \in \mathrm{BL}_m(\Rd)$.

\section{Road map}
Let's start with a road map for the proof of Theorem~\ref{thm:tps-representer}. The fundamental idea will be to ``project out'' the null space $\mc{N}$ of $J_m^d(f)$: that is, decompose functions $f \in \BL_m(\Rd)$ into the sum of two orthogonal parts $$f = Pf + [I - P]f, Pf \in \mc{H}_0, [I - P]f \in \mc{H}_1$$ where $P$ is the projection of $f \in \BL_m(\Rd)$ with respect to a (hitherto undetermined) inner product. This is useful because the space $\mc{H}_1$ -- which is the orthogonal subspace to $\mc{H}_0$, again with respect to a hitherto undetermined inner product -- is an RKHS; letting $\{\eta_{x}^{(m)}: x \in \Rd\}$ be the representers of evaluation in this Hilbert space, we can use standard arguments in the theory of RKHS to show that $[I - P]f_{\lambda} \in \mathrm{span}\{\eta_{x_1}^{(m)},\ldots, \eta_{x_n}^{(m)}\}$. This is what will end up happening in Proposition~\ref{prop:tps-representer}. Finally we will use an explicit representation of the reproducing kernel $\eta^{(m)}$ to rewrite the solution in terms of $E_{x_1},\ldots,E_{x_n}$, giving us the desired result. 

\section{Direct sum RKHS}
In order to execute this plan we need to define a suitable inner product on $\mathrm{BL}_m(\Rd)$ with which to project out polynomials.  To that end let $\{z_0,\ldots,z_M\}$ be a collection of $\mc{N}$-unisolvent points, meaning
$$ \sum_{\nu = 1}^{M} c_{\nu} \phi_{\nu}(z_k) = 0 ~~\textrm{for all $k$} \Longrightarrow  c = 0.$$
Now introduce a second semi-inner product over $\mathrm{BL}_{m}(\Rd)$: 
\begin{equation}
\label{eqn:semi-inner-product}
(u,v)_{0} := \sum_{j = 1}^{M} u(z_j) v(z_j).
\end{equation}
Then $\dotp{u}{v} := (u,v)_0 + (u,v)_{m}$ is an inner product on $\mathrm{BL}_m(\Rd)$, and
$$\mc{H} = \{f \in \mathrm{BL}_{m}(\Rd):  \dotp{f}{f} < \infty\}$$ is an RKHS.

Now we introduce the spaces $\mc{H}_0, \mc{H}_1$ alluded to above. $\mc{H}_0$ is simply the null space $\mc{N}$ of $J_m^d(\cdot)$, equipped with the inner product $(\cdot,\cdot)_0)$ defined in~\eqref{eqn:semi-inner-product}, and $\mc{H}_1$ is the orthogonal complement:
$$\mc{H}_1 = \{f \in \mathrm{BL}_m(\Rd), f(z_k) = 0~\textrm{for each}~z_k\}.$$
These spaces are orthogonal in the sense that $\dotp{f_0}{f_1} = 0$ for $f_0 \in \mc{H}_0$ and $f_1 \in \mc{H}_1$. Additionally both are RKHS, and the reproducing kernel $\eta(x,y)$ of $\mc{H}$ can be decomposed as
\begin{equation*}
\eta(x,y) = \eta^{(0)}(x,y) + \eta^{(m)}(x,y),
\end{equation*}
where $\eta^{(0)}$ is the reproducing kernel of $\mc{H}_0$ and $\eta^{(m)}$ the reproducing kernel of $\mc{H}_1$. So we can write $\mc{H} = \mc{H}_0 \oplus \mc{H}_1$ as a direct sum RKHS.

As mentioned, we prove Theorem~\ref{thm:tps-representer} by establishing a representation of the solution in terms of $\eta^{(m)}$ and then rewriting this as a function of $E$. In this second step we use an explicit relation between the kernel and semi-kernel in terms of $P$. Taking $\{\varphi_1,\ldots,\varphi_M\}$ to be an orthobasis of $\mc{H}_0$ we can write the projection operator as $Pf(x) = \sum_{\nu} (\varphi_{\nu},f)_0 \varphi_{\nu}$, and for simplicity taking $\{\varphi_1,\ldots,\varphi_M\}$ to be the canonical basis (so that $\varphi_{\nu}(z_k) = \delta_{\nu k}$) this becomes
\begin{equation*}
	Pf(x) = \sum_{\nu} f(z_{\nu}) \varphi_{\nu}(x).
\end{equation*}
%Write $(P^{(x)}E)$ for the projection of $E$, treated as a function of its first argument, into $\mc{N}$:
%\begin{equation*}
%	(P^{(x)}E)(u,t) := \sum_{\nu} (\varphi_{\nu},E_u)_0 \varphi_{\nu}(t) = \sum_{\nu} E_{u}(z_{\nu}) \varphi_{\nu}(t)
%\end{equation*}
%Likewise $P^{(y)}E$ for the projection of $E$ treated as a function of its second argument. Notice that $P^{(x)}E$ is a polynomial in $y$ and $P^{(x)}P^{(y)}E$ is a polynomial in $x$ and $y$. 

\begin{theorem}
	\label{lem:rk-greens}
The reproducing kernel of $\mc{H}_1$, evaluated at a given $x,y$, is given by
\begin{equation}
	\label{eqn:rk}
	\eta^{(m)}(x,y) = E(x,y) - \sum_{\nu} E(x,z_{\nu}) \varphi_{\nu}(y) - \sum_{\nu'} E(z_{\nu'},y) \varphi_{\nu'}(x) + \sum_{\nu,\nu'} \varphi_{\nu}(x) \varphi_{\nu'}(y) E(z_{\nu},z_{\nu'}).
\end{equation}
\end{theorem}

\paragraph{Proof.}
This result and proof is due to \citet{meinguet1979}. I will go fast without really doing the argument justice, and you should consult~\citet{meinguet1979} for missing details.

We begin from the definition of the reproducing kernel of $\mc{H}_1$, that it is the Riesz representer of $\delta_x$ in $\mc{H}_1$ and so
\begin{equation*}
	(\eta_x^{(m)},f) = f(x), \quad \textrm{for all $f \in \mc{H}_1$}
\end{equation*}

Let $\mc{D}$ be the Schwartz space of distributions, i.e. continuous linear functionals on the set of smooth and compactly supported functions $C_c^{\infty}(\Rd) = \mc{D}'$. By definition $[I - P]f \in \mc{H}_1$ and so
\begin{equation}
	\label{pf:rk-greens-1}
	f(x) - \sum_{\nu = 1}^{M} f(z_{\nu}) \varphi_{\nu}(x) = ([I - P]f, \eta^{(m)})_m = (f,\eta^{(m)})_m.
\end{equation}
Now we make use of repeated application of integration by parts -- keeping in mind that $f \in \mc{D}'$ is compactly supported -- to deduce that
\begin{equation*}
	(\eta^{(m)},f) = \bigl((-1)^m \Delta^m \eta^{(m)},f\bigr)
\end{equation*}
where $(\cdot,\cdot)$ is the duality pairing of $\mc{D},\mc{D}'$. Rewriting the left hand side of~\eqref{pf:rk-greens-1} in terms of this duality pairing, we obtain the differential equation
\begin{equation}
	\label{pf:rk-greens-2}
	(-1)^m \Delta^m \eta_x^{(m)} = \delta_{x} - \sum_{\nu = 1}^{M} \varphi_{\nu}(x) \delta_{z_\nu},
\end{equation}
which has to be interpreted in the distributional sense of operators acting on $\mc{D}'$.

Now from the definition of $E$ we can easily get a distribution $H_x$ satisfying~\eqref{pf:rk-greens-2},
\begin{equation*}
	H_x = E_x - \sum_{\nu = 1}^{M} \varphi_{\nu}(x) E_{z_{\nu}}.
\end{equation*}
At this point we will use an important result without proof: $H_x \in \mc{H}$. To get a sense of why this is non-trivial, note that the same statement does not hold true for $E_x$. However, once we take for granted that $H_x \in \mc{H}$ the proof is almost finished; clearly $[I - P]H_x \in \mc{H}_1$ and in fact $[I - P]H_x$ is the unique function in $\mc{H}_1$ for which~\eqref{pf:rk-greens-2} is satisfied (since $P$ projects to the null space of $\Delta^m$.) So $\eta^{(m)} = [I - P]H_x$. Solving for $[I - P]H_x$ in terms of $E$ then gives the desired result.

\section{Representer theorem for TPS}
Now we have everything we need to represent $f_{\lambda}$ in terms of the reproducing kernel of $\mc{H}^{1}$. 
\begin{proposition}
	\label{prop:tps-representer}
	The solution $f_{\lambda}$ to problem~\eqref{eqn:tps} can be written as
	\begin{equation}
	\label{eqn:tps-representer}
		f_{\lambda} = \sum_{\nu = 1}^{M} c_{\nu} \phi_{\nu} + \sum_{i = 1}^{n} d_i \eta^{(m)}_{x_i}.
	\end{equation}
	Thus, letting ${\bf Q} \in \Reals^{n \times n}$ have entries ${\bf Q}_{ij} = \eta^{(m)}(x_i,x_j)$, the problem~\eqref{eqn:tps} can be recast as the generalized ridge problem
	\begin{equation}
		\label{eqn:generalized-ridge}
		\min_{c,d} \|y - {\bf \Phi} c - {\bf Q} d\|^2 + \lambda d^{\top} {\bf Q} d \quad \textrm{subject to ${\bf \Phi}^{\top}d = 0$.}
	\end{equation}
\end{proposition}

\paragraph{Proof.}
Decompose $f_{\lambda} = Pf_{\lambda} + [I - P]f_{\lambda}$. Observe that since the smoothness functional $J_m^d(f_{\lambda}) = (f_{\lambda},f_{\lambda})_m = ([I - P]f_{\lambda}, [I - P]f_{\lambda})$ we can rewrite the objective in~\eqref{eqn:tps}, evaluated at its minimizer $f_{\lambda}$, as
\begin{equation*}
	\frac{1}{n}\sum_{i = 1}^{n}\Bigl(y_i - Pf_{\lambda}(x_i) - [I - P]f_{\lambda}(x_i)\Bigr)^2 + \lambda J_d^{m}([I - P]f_{\lambda}).
\end{equation*}
From here the analyis is completely standard for the theory of reproducing kernels. Further decompose $[I - P]f_{\lambda} = f_{\lambda}^{(\pi)} + f_{\lambda}^{\perp}$, where $f_{\lambda}^{(\pi)} \in \mathrm{span}(\eta_{x_i}^{(m)})$ and $f_{\lambda}^{\perp}$ belongs to the orthogonal complement. Since $\eta_{x_i}^{(m)}$ is the representer of evaluation for $\mc{H}_1$ and since $f_{\lambda}^{(\pi)}$ and $f_{\lambda}^{\perp}$ are orthogonal it follows that 
$$f_{\lambda}^{\perp}(x_i) = (f_{\lambda}^{\perp}, \eta_{x_i}^{(m)})_m = 0.$$ 
Also, again using the orthogonality of $f_{\lambda}^{(\pi)}$ and $f_{\lambda}^{\perp}$,
$$J_d^{m}\bigl([I - P]f_{\lambda}\bigr) = J_d^{m}(f_\lambda^{(\pi)}) + J_d^{m}(f_\lambda^{\perp}) \geq J_d^{m}(f_\lambda^{(\pi)}),$$ with equality only if $f_\lambda^{\perp} = 0$. We conclude that $f_{\lambda}^{\perp} = 0$, from which~\eqref{eqn:tps-representer} follows. 

Finally, to complete the proof of Proposition~\ref{prop:tps-representer} we need to show the equivalence between the original problem and~\eqref{eqn:generalized-ridge}. First of all, noting that $f_{\lambda}(x_i) = {\bf \Phi}_{i\cdot} c + {\bf Q}_{i \cdot} d$ and $J_d^{m}([I - P]f_{\lambda}) = d^{\top} {\bf Q} d$,  we see that~\eqref{eqn:tps} can be rewritten as 
\begin{equation*}
\min_{c,d} \|y - {\bf \Phi} c - {\bf Q} d\|^2 + \lambda d^{\top} {\bf Q} d,
\end{equation*}
which is~\eqref{eqn:generalized-ridge} without the constraint ${\bf \Phi}^{\top}d = 0$. To see why this constraint additionally holds, we look at the normal equations for~\eqref{eqn:generalized-ridge}, which are
\begin{equation}
	\label{eqn:normal}
	\begin{aligned}
	{\bf \Phi}^{\top} y &=  {\bf \Phi}^{\top} {\bf Q} \hat{d} + {\bf \Phi}^{\top} {\bf \Phi} \hat{c} \\
	{\bf Q} y & = \lambda {\bf Q} \hat{d}  + {\bf Q}{\bf Q} \hat{d} + {\bf Q} {\bf \Phi} \hat{c}.
	\end{aligned}
\end{equation}
Since $\eta^{(m)}$ is a positive definite kernel ${\bf Q}$ is invertible. So we may rewrite the second equation in~\eqref{eqn:normal} as $y = \lambda \hat{d} + {\bf Q} \hat{d} + {\bf \Phi} \hat{c}$, and plugging this into the first equation in~\eqref{eqn:normal} yields
\begin{equation*}
	\lambda {\bf \Phi}^{\top} \hat{d} = 0.
\end{equation*}

\section{Representation in terms of Green's function}
Lemma~\ref{lem:rk-greens} gives the explicit relationship between $\eta^{(m)}$ and the Green's function $E$. Applying this to $\sum_{i = 1}^{n} \hat{d}_i \eta_{x_i}^{(m)}$ gives the following expression:
\begin{equation*}
	\sum_{i = 1}^{n} \hat{d}_i \eta_{x_i}^{(m)}(t) = \sum_{i = 1}^{n} \hat{d}_i E_{x_i}(t) - \sum_{i = 1}^{n} \sum_{\nu} \hat{d}_i E(x_i,z_{\nu}) \varphi_{\nu}(t) + \sum_{i = 1}^{n} \hat{d}_i \sum_{\nu'} E_{t}(z_{\nu'}) \varphi_{\nu}(x_i) + \sum_{i = 1}^{n} \hat{d}_i \sum_{\nu,\nu'} \varphi_{\nu}(t) \varphi_{\nu'}(x_i) E(z_{\nu},z_{\nu'}).
\end{equation*}
On the right hand side the second and fourth terms are polynomial in $t$. On the other hand, $\hat{d}$ lies in the kernel of ${\bf \Phi}^{\top}$ and so the third term is $0$. We conclude that $\sum_{i = 1}^{n} \hat{d}_i \eta_{x_i}^{(m)} = \sum_{i = 1}^{n} \hat{d}_i E_{x_i} + p$ for some polynomial $p$, and consequently for some $\wt{c}$,
\begin{equation}
	\label{pf:thm1-1}
	f_{\lambda} = \sum_{\nu = 1}^{M} \tilde{c}_{\nu} \phi_{\nu} + \sum_{i = 1}^{n} \hat{d}_i E_{x_i}.
\end{equation}

\section{Proof of Theorem~\ref{thm:tps-representer}}
Combining~\eqref{pf:thm1-1} and Proposition~\ref{prop:tps-representer} shows that the solution to~\eqref{eqn:tps} is equal to
\begin{equation}
	\label{pf:thm1-2}
	\min_{c,d} \|y - {\bf \Phi} c - {\bf E} d\|^2 + \lambda \hspace{2 pt} J_d^m\biggl(\sum_{i = 1}^{n} d_i E_{x_i}\biggr) \quad \textrm{subject to ${\bf \Phi}^{\top}d = 0$.}
\end{equation}
It remains only to show that the penalty in~\eqref{pf:thm1-2} has the finite-dimensional representation as written in~\eqref{eqn:tps-representer}. This can be seen by another application of Lemma~\ref{lem:rk-greens}: for any $d$ such that ${\bf \Phi}^{\top} d = 0$,
\begin{align*}
	J_d^m\biggl(\sum_{i = 1}^{n} d_i E_{x_i}\biggr) & = J_d^m\biggl(\sum_{i = 1}^{n} d_i \eta_{x_i}^{(m)}\biggr) \\
	& = \sum_{i,j = 1}^{n} c_i c_j \eta^{(m)}(x_i,x_j) \\
	& = \sum_{i,j = 1}^{n} c_i c_j \biggl\{E(x_i,x_j) -  \sum_{\nu} E(x_i,z_{\nu}) \varphi_{\nu}(x_j) - \sum_{\nu'} E(z_{\nu'},x_j) \varphi_{\nu'}(x_i) + \sum_{\nu,\nu'} \varphi_{\nu}(x_i) \varphi_{\nu'}(x_j) E(z_{\nu},z_{\nu'}) \biggr\} \\
	& = \sum_{i,j = 1}^{n} c_i c_j E(x_i,x_j),
\end{align*}
where the first equality follows since $\eta_{x_i}^{m}$ and $E_{x_i}$ differ by a polynomial.



%Finally we have that
%\begin{equation*}
%f_{\lambda}(x_i) = Pf_{\lambda}(x_i) + f_{\lambda}^{\pi}(x_i) = (f_{\lambda},\eta_{x_i}^{(0)})_0 + %(f_{\lambda}^{\pi},\eta_{x_i}^{(m)})_m = \sum_{}
%\end{equation*}

%where $f_{\lambda}^{\perp}$ is orthogonal to everything of the form $\{\sum_{\nu = 1}^{M} \alpha_{\nu} \phi_{\nu} + \sum_{i = 1}^{n} \beta_i \eta_{x_i}^{(m)}\}$. This means
%\begin{equation*}
%\dotp{f_{\lambda}^{\perp}}{\eta_{x_i}^{(m)}} = 0~\textrm{for all $i = 1,\ldots,n$}~~\textrm{and}~~\dotp{f_{\lambda}^{\perp}}{\phi_j} = 0~\textrm{for all $j = 1,\ldots,M$.}
%\end{equation*}
%The second of these properties, along with the fact that 
%$$0 = \dotp{f_{\lambda}^{\perp}}{\phi_{\nu}} = (f_{\lambda}^{\perp},\phi_{\nu})_0 = \sum_{j = 1}^{M} f_{\lambda}^{\perp}(z_j) %\phi_{\nu}(z_j)$$ 
%implies that $f_{\lambda}^{\perp}(z_j) = 0$ for all $z_j$. Since $\eta^{(m)}$ is the representer of evaluation in $\mc{H}^1$, and we have just shown $f_{\lambda}^{\perp} \in \mc{H}_1$, the first of these properties implies that
%\begin{equation}
%f_{\lambda}^{\perp}(x_i) = (f_{\lambda}^{\perp},\eta_{x_i}^{(m)})_{m} =\dotp{f_{\lambda}^{\perp}}{\eta_{x_i}^{(m)}} = 0.
%\end{equation} 
%Therefore,
%\begin{equation*}
%\begin{aligned}
%(f_{\lambda},f_{\lambda})_m & = (f_{\lambda,\pi},f_{\lambda,\pi})_m + (f_{\lambda}^{\perp},f_{\lambda}^{\perp})_m + %2(f_{\lambda,\pi}, f_{\lambda}^{\perp})_m \\
%& = (f_{\lambda,\pi},f_{\lambda,\pi})_m + (f_{\lambda}^{\perp},f_{\lambda}^{\perp})_m - 2(f_{\lambda,\pi}, f_{\lambda}^{\perp})_0 \\
%& = (f_{\lambda,\pi},f_{\lambda,\pi})_m + (f_{\lambda}^{\perp},f_{\lambda}^{\perp})_m.
%\end{aligned}
%\end{equation*}
%We conclude that $f_{\lambda} = f_{\lambda,\pi}$. Finally, 
%\begin{equation*}
%	f_{\lambda,\pi}(x_i) = \dotp{f_{\lambda}}{\eta_{x_i}} = (f_{\lambda}, \eta_{x_i}^{(0)})_0 + (f_{\lambda}, \eta_{x_i}^{(m)})_m = \sum_{\nu = 1}^{M} c_\nu \phi_{\nu}(x_i) + \sum_{i = 1}^{n} d_i \eta^{(m)}(x_i,x_j),
%\end{equation*}
%and $(f_{\lambda,\pi},f_{\lambda,\pi})_{m} = \sum_{i,j = 1}^{n} d_i d_j \eta^{(m)}(x_i,x_j)$. Written in matrix form this is just~\eqref{eqn:generalized-ridge}. \qed

%At this point we could reasonably say our work is done. We have related the domain over which TPS is conducted to an RKHS, and used this relationship to prove a representer theorem. However the operation of ``projecting out polynomials'' in~\eqref{eqn:rk}, while not too difficult to grasp mathematically, makes the closed form expression for the kernel somewhat unwieldy. The original representer Theorem~\ref{thm:tps-representer} is simpler and does not depend on $z_1,\ldots,z_M$. 


\end{document}