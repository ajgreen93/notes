\documentclass{article}

% Packages
\usepackage[utf8]{inputenc} % allow utf-8 input
\usepackage[T1]{fontenc}    % use 8-bit T1 fonts
\usepackage{booktabs}       % professional-quality tables
\usepackage{nicefrac}       % compact symbols for 1/2, etc.
\usepackage{microtype}      % microtypography
\usepackage{times}             % times font
\usepackage{mathrsfs}      % Added by Alden, for script font.
\usepackage{ragged2e}     % Added by Alden, for no indent.
\usepackage{parskip}        % Added by Alden, for skips between paragraphs.

\usepackage[round]{natbib}
\usepackage{amssymb,amsmath,amsthm,bbm}
\usepackage[margin=1in]{geometry}
\usepackage{verbatim,float,url,dsfont}
\usepackage{graphicx,subcaption,psfrag} % Alden changed subfigure to subcaption
\usepackage{algorithm,algorithmic}
\usepackage{mathtools}
\usepackage[shortlabels]{enumitem}       % Alden added shortlabels option
\usepackage[colorlinks=true,citecolor=blue,urlcolor=blue,linkcolor=blue]{hyperref}
\usepackage{multirow}

% Theorems and such
\newtheorem{theorem}{Theorem}
\newtheorem{lemma}{Lemma}
\newtheorem{corollary}{Corollary}
\newtheorem{proposition}{Proposition}
\theoremstyle{definition}
\newtheorem{remark}{Remark}
\newtheorem{definition}{Definition}
\newtheorem{example}{Example} % Added by Alden
\newtheorem{conjecture}{Conjecture} % Added by Alden

% Assumption
\newtheorem*{assumption*}{\assumptionnumber}
\providecommand{\assumptionnumber}{}
\makeatletter
\newenvironment{assumption}[2]{
  \renewcommand{\assumptionnumber}{Assumption #1#2}
  \begin{assumption*}
  \protected@edef\@currentlabel{#1#2}}
{\end{assumption*}}
\makeatother

% Widebar
\makeatletter
\newcommand*\rel@kern[1]{\kern#1\dimexpr\macc@kerna}
\newcommand*\widebar[1]{%
  \begingroup
  \def\mathaccent##1##2{%
    \rel@kern{0.8}%
    \overline{\rel@kern{-0.8}\macc@nucleus\rel@kern{0.2}}%
    \rel@kern{-0.2}%
  }%
  \macc@depth\@ne
  \let\math@bgroup\@empty \let\math@egroup\macc@set@skewchar
  \mathsurround\z@ \frozen@everymath{\mathgroup\macc@group\relax}%
  \macc@set@skewchar\relax
  \let\mathaccentV\macc@nested@a
  \macc@nested@a\relax111{#1}%
  \endgroup
}
\makeatother

% Min and max
\newcommand{\argmin}{\mathop{\mathrm{argmin}}}
\newcommand{\argmax}{\mathop{\mathrm{argmax}}}
\newcommand{\minimize}{\mathop{\mathrm{minimize}}}
\newcommand{\st}{\mathop{\mathrm{subject\,\,to}}}
\DeclareMathOperator*{\esssup}{ess\,sup} % Added by Alden

% Shortcuts
\def\R{\mathbb{R}}
\def\C{\mathbb{C}}
\def\E{\mathbb{E}}
\def\P{\mathbb{P}}
\def\T{\mathsf{T}}
\def\Cov{\mathrm{Cov}}
\def\Var{\mathrm{Var}}
\def\half{\frac{1}{2}}
\def\tr{\mathrm{tr}}
\def\df{\mathrm{df}}
\def\dim{\mathrm{dim}}
\def\col{\mathrm{col}}
\def\row{\mathrm{row}}
\def\nul{\mathrm{null}}
\def\rank{\mathrm{rank}}
\def\nuli{\mathrm{nullity}}
\def\spa{\mathrm{span}}
\def\sign{\mathrm{sign}}
\def\supp{\mathrm{supp}}
\def\diag{\mathrm{diag}}
\def\aff{\mathrm{aff}}
\def\conv{\mathrm{conv}}
\def\dom{\mathrm{dom}}
\def\hy{\hat{y}}
\def\hf{\hat{f}}
\def\hmu{\hat{\mu}}
\def\halpha{\hat{\alpha}}
\def\hbeta{\hat{\beta}}
\def\htheta{\hat{\theta}}
\def\cA{\mathcal{A}}
\def\cB{\mathcal{B}}
\def\cD{\mathcal{D}}
\def\cE{\mathcal{E}}
\def\cF{\mathcal{F}}
\def\cG{\mathcal{G}}
\def\cK{\mathcal{K}}
\def\cH{\mathcal{H}}
\def\cI{\mathcal{I}}
\def\cL{\mathcal{L}}
\def\cM{\mathcal{M}}
\def\cN{\mathcal{N}}
\def\cP{\mathcal{P}}
\def\cS{\mathcal{S}}
\def\cT{\mathcal{T}}
\def\cW{\mathcal{W}}
\def\cX{\mathcal{X}}
\def\cY{\mathcal{Y}}
\def\cZ{\mathcal{Z}}

%%% Begin Alden's additions
\newcommand{\Ebb}{\mathbb{E}}
\newcommand{\Pbb}{\mathbb{P}}
\newcommand{\dotp}[2]{\langle #1, #2 \rangle}
\newcommand{\wt}[1]{\widetilde{#1}}
\newcommand{\wh}[1]{\widehat{#1}}
\newcommand{\mc}[1]{\mathcal{#1}}
\newcommand{\Reals}{\mathbb{R}} % Same thing as Ryan's \R
\newcommand{\Rd}{\Reals^d}
\newcommand{\wb}[1]{\widebar{#1}}
\newcommand{\floor}[1]{\left\lfloor #1 \right\rfloor}
\newcommand{\ceil}[1]{\left\lceil #1 \right\rceil}
\newcommand{\1}{\mathbf{1}}
\newcommand{\bj}{{\bf j}}
\newcommand{\restr}[2]{\ensuremath{\left.#1\right|_{#2}}}
\newcommand{\TV}{\mathrm{TV}}

\DeclareFontFamily{U}{mathx}{\hyphenchar\font45}
\DeclareFontShape{U}{mathx}{m}{n}{<-> mathx10}{}
\DeclareSymbolFont{mathx}{U}{mathx}{m}{n}
\DeclareMathAccent{\wc}{0}{mathx}{"71}


%%% End Alden's Additions
\usepackage{lmodern}
\usepackage{xcolor}
\usepackage{diagbox}
\usepackage{xr-hyper}

\newcommand{\bx}{\boldsymbol{x}}
\newcommand{\bz}{\boldsymbol{z}}
\newcommand{\bv}{\boldsymbol{v}}
\newcommand{\bw}{\boldsymbol{w}}
\newcommand{\bg}{\boldsymbol{g}}
\newcommand{\bq}{\boldsymbol{q}}
\newcommand{\bX}{\boldsymbol{X}}
\newcommand{\bbeta}{\boldsymbol{\beta}}
\newcommand{\bSigma}{\boldsymbol{\Sigma}}
\newcommand{\bTheta}{\boldsymbol{\Theta}}
\newcommand{\bS}{\boldsymbol{S}}
\newcommand{\bT}{\boldsymbol{T}}
\newcommand{\bB}{\boldsymbol{B}}
\newcommand{\bG}{\boldsymbol{G}}
\newcommand{\bQ}{\boldsymbol{Q}}
\newcommand{\bP}{\boldsymbol{P}}
\newcommand{\bR}{\boldsymbol{R}}
\newcommand{\bI}{\boldsymbol{I}}
\newcommand{\bZ}{\boldsymbol{Z}}
\newcommand{\Complex}{\mathbb{C}}

\newcommand{\interior}{\mathrm{int}}
\newcommand{\range}{\mathrm{range}}

\newcommand{\convweak}{\Rightarrow}

\title{ {\bf Analysis of ridge using basic inequality} }

\begin{document}
	
	\maketitle
	\RaggedRight
	We have fixed design $X \in \R^{n \times p}$, and response $y \sim N_n(X\beta^{\ast}, \sigma^2 I)$; we'll let $\sigma^2 = 1$ for convenience. We assume $n > p$, that $X$ is full rank, and that $\frac{1}{n} X^{\top} X$ is within a multiplicative factor of the identity:
	$$
	\|\frac{1}{n} X^{\top} X\|_{\mathrm{op}}, \|(\frac{1}{n} X^{\top} X)^{-1}\|_{\mathrm{op}} \leq C,
	$$
	where here and throughout $C$ is a universal constant that can change from line to line. This means that $\|Xb\|_2 \leq C \sqrt{n} \|b\|_2$ and likewise $\|b\|_2 \leq C/\sqrt{n} \|Xb\|_2$, and this fact will be used implicitly. 
	
	We assume $\|\beta^{\ast}\|_2 \leq b$ and estimate $\beta^{\ast}$ via ridge regression:
	$$
	\hat{\beta} = \argmin \|y - X\beta\|_2^2 + \lambda \|\beta\|_2^2.
	$$
	We are thinking of $b = O(1)$, in which case ridge may not offer much improvement over OLS, at least at the level of rates. Certainly, we will want results that apply to the case $\lambda = 0$.
	
	We are interested in the excess risk of $\hat{\beta}$: for $\tilde{y} \sim N_n(X\beta^{\ast}, \sigma^2 I)$ independent of $y$:
	$$
	\frac{1}{n}\E\Big[\|\tilde{y} - X\hat{\beta}\|_2^2 - \|\tilde{y} - X\beta^{\ast}\|_2^2\Big] = \frac{1}{n} \E\|X(\hat{\beta} - \beta^{\ast})\|_2^2.
	$$
	To get a bound on the error term here, we first apply a basic inequality for penalized estimators:
	$$
	\|y - X\hat{\beta}\|_2^2 + \lambda \|\hat{\beta}\|_2^2 \leq \|y - X\beta^{\ast}\|_2^2 + \lambda \|\beta^{\ast}\|_2^2,
	$$
	and therefore
	\begin{align*}
	\|X(\hat{\beta} - \beta^{\ast})\|_2^2 
	& \leq 2 \langle \hat{\beta} - \beta^{\ast}, X^{\top}\epsilon \rangle + \lambda(\|\beta^{\ast}\|_2^2 - \|\hat{\beta}\|_2^2) \\
	& \leq 2 \|\hat{\beta} - \beta^{\ast}\|_2 \cdot \sup_{\beta \in B_d(1)}\langle \beta, X^{\top}\epsilon \rangle + \lambda(\|\beta^{\ast}\|_2^2 - \|\hat{\beta}\|_2^2) \\
	& \leq 2 \|\hat{\beta} - \beta^{\ast}\|_2 \cdot \sup_{\beta \in B_d(1)}\langle \beta, X^{\top}\epsilon \rangle +  2 \lambda b \|\hat{\beta} - \beta^{\ast}\|_2.
	\end{align*}
	The second inequality above is obvious, the final inequality follows since either $\|\hat{\beta}\|^2 \geq b$ and therefore the second term is non-positive, or $\|\hat{\beta}\|^2 \leq b$ and then
	$$
	\|\beta^{\ast}\|_2^2 - \|\hat{\beta}\|_2^2 = \dotp{\beta^{\ast} + \hat{\beta}}{\beta^{\ast} - \hat{\beta}} \leq \|\beta^{\ast} + \hat{\beta}\|_2 \|\beta^{\ast} - \hat{\beta}\|_2  \leq  2b \|\beta^{\ast} - \hat{\beta}\|_2.
	$$
	For the Gaussian complexity term, we use the fact that $B_d(1)$ can be covered by $O((1/\delta)^d)$ balls of radius $\delta$. Let $\mc{C}$ denote such a covering. Of course, for each $\beta \in B_d(1)$, $\beta^{\top} X^{\top} \epsilon \sim N(0, \|X\beta\|_2^2)$ and therefore
	$$
	\P(\beta^{\top} X^{\top} \epsilon  \geq \phi ) \leq C \exp(-c \phi^2/n),
	$$
	for some universal constants $C$ and $c$. Therefore,
	$$
	\sup_{\beta \in B_d(1)} \dotp{\beta}{X^{\top} \epsilon} \leq \sup_{\beta \in \mc{C}} \dotp{\beta}{X^{\top} \epsilon} + \sup_{\beta \in B_d(1)} \inf_{\beta' \in \mc{C}} \|X(\beta - \beta')\|_2 \|\epsilon\|_2 \leq C\Big(n \delta + \phi\Big),
	$$ 
	with probability at least $1 - \delta^{-d} \exp(-c \phi^2/n)$.
	Taking $\delta = \sqrt{d/n}$ and $\phi = 2/c\sqrt{n \log(1/\delta)} = C\sqrt{n d \log n}$, we have that 
	$$
	\sup_{\beta \in B_d(1)} \dotp{\beta}{X^{\top} \epsilon} \leq C \sqrt{n d \log n},
	$$
	with probability at least $1 - Cn^{-2}$. Note: the $\sqrt{\log n}$ factor is loose and can be removed by tighter analysis, as described e.g. in Example 5.23 of the Wainwright textbook.
	
	As a result, for any $\lambda \leq \sqrt{n}$,
	\begin{align*}
	\|X(\hat{\beta} - \beta^{\ast})\|_2^2  
	& \leq C \|\hat{\beta} - \beta^{\ast}\|_2 \sqrt{n d \log n} + 2 \lambda b \|\hat{\beta} - \beta^{\ast}\|_2 \\
	& \leq C\|X(\hat{\beta} - \beta^{\ast})\|_2 \sqrt{d \log n} + 2 b \|X(\hat{\beta} - \beta^{\ast})\|_2.
	\end{align*}
	Rearranging terms gives
	\begin{equation*}
		\|X(\hat{\beta} - \beta^{\ast})\|_2 \leq C\Big(\sqrt{d \log n} + 2 b\Big),
	\end{equation*}
	and therefore,
	\begin{equation*}
		\frac{1}{n} \|X(\hat{\beta} - \beta^{\ast})\|_2^2 \leq \frac{C}{n}\Big(d \log n + 2 b^2\Big),
	\end{equation*}
	with probability $1 - C n^{-2}$. This recovers the right dependence on $d,n$, up to the loose factor of $\log n$ mentioned before. 
\end{document}